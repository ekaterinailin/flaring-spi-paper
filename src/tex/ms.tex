% Define document class
\documentclass[twocolumn]{aastex631}
\usepackage{showyourwork}

% Begin!
\begin{document}

% Title
\title{Searching for flaring star-planet interactions in star-planet systems in Kepler and TESS}

% Author list
\author{Ekaterina Ilin}
\author{Katja Poppenh\"ager}

% Abstract with filler text
\begin{abstract}
    Star-planet system we have found so far are nothing like the Solar System. Many stars host planets in orbits much closer than Mercury, where it takes them mere days to complete a full revolution. In some of these systems, the planets are so close that they plow through the extended magnetosphere of the star, and are expected to trigger flares on their surfaces. So far, however, these flaring star-planet interactions escape our detection methods. If found, the planets and the flare they cause become localized probes of the parts of the stellar magnetosphere that extends from the location of one to the other. A measurement of these interactions constrains mass loss from the star, and potentially also the magnetic field of the planet. 

    In this work, we searched star-planet systems observed by Kepler and TESS for flaring star-planet interactions. We compiled a catalog of all flaring star-planet systems known to date, and characterized the detected flares detected. We then searched for flares in excess of the intrinsic flaring of the star, i.e., flares triggered by the interaction with the planet in phase with its orbit. 

    We find XXXX star-planet systems with > XXXX flares. For XXXX of them, we can calculate the expected amplitude of interaction based on the system's magnetic field strength, orbital parameters, rotation and planet radius. We find that the trend of higher interaction signal agrees with the expected strength, but also shows signs of intermittency, both for individual systems, and in comparison between ones with similar expected amplitude. HIP 67522, a 17 Myr young Sun with a Jupiter in a 9 day orbit, shows both one of the highest expected and the highest measured interaction amplitudes, albeit only at the $2.5\sigma$ level. 
\end{abstract}

% Main body with filler text
\section{Introduction}
\label{sec:intro}
Star-planet system we have found so far are nothing like the Solar System.

What is SPI?

Why look for SPI? Use planets as probes stellar magnetic fields at a defined orbit, e.g., stellar wind conditions. SPI can only occur if the planet is orbiting within the Alfv\'en radius of the star for a sizable period of time. The Alfv\'en radius is the radius at which the stellar wind velocity, which increases with increasing distance from the star because of XXXX, exceed the Alfv\'en velocity of the magnetized plasma. Beyond this radius, this plasma is disconnected from the star. Alfv\'en waves cannot propagate back to the star because the wind is carrying the plasma away faster than these waves can travel.

Stretch and break


Alfven wing



\section{Data}
\label{sec:data}
\subsection{Star-planet systems}


ction with the Planetary Systems Composite Parameters Table (PSCP)\footnote{ \url{https://exoplanetarchive.ipac.caltech.edu/cgi-bin/TblView/nph-tblView?app=ExoTbls&config=PSCompPars}} from which we removed controversial planet detections (``pl\_controv\_flag'' set to 1), i.e. detections with existing literature questioning the result. The catalog then contained XXXX unique systems.

We complemented the PSCP with the TESS TOI Table~\citep[TT][]{guerrero2021arxiv}\footnote{\url{https://tev.mit.edu/data/collection/193/}}. We filtered the catalog by requiring the confirmed planet (CP) or known planet (KP) flags to be set in the ``TOI Disposition''. This added XXXX star-planet systems to the sample.

\subsection{Kepler and TESS photometry}
Between 2009 and 2013, the Kepler space telescope~\citep{koch2010kepler} nearly continuously observed a patch of the sky in the Cygnus-Lyra region. Each of the 18 observing Quarters contains nearly uninterrupted $\sim 90$ days of observations, totaling over 100000 stars monitored in 2-min cadence. 
% 100000 from here: https://exoplanetarchive.ipac.caltech.edu/docs/KeplerMission.html

The Transiting Exoplanet Survey Satellite~(TESS,~\cite{ricker2015transiting}) is an all-sky mission that began operations in 2018, completed two sky scans until summer 2022. It is observing at the time of writing, collecting nearly continuous photometric time series in the 600-1000 nm band for $\sim 27 days$ in each observing Sector. About $200\,000$ stars have been observed in 2-min cadence in the first two years of operations with about 20000 targets per Sector. Out of these, from Sector 27 on, 1000 targets were observed at even higher 20-s cadence in each Sector. 
% 20000 per sector does not add up, because of stars observed in multiple sectors, the 20000 and 1000 figures can be found here: https://tess.mit.edu/observations/target-lists/
% extended TESS mission https://heasarc.gsfc.nasa.gov/docs/tess/extended.html

Based on the filtered PSCP and TT tables, we queried the full Kepler archive (quarters Q0-Q17, Data Release 25) and most recent TESS catalog (July 2022) for their respective 1-min and 2-min/20-s cadence light curves, accounting for systems that appeared in both catalogs. In total, we obtained XXXX Kepler light curves and XXXX TESS light curves. XXXX systems were observed only by the primary Kepler mission, XXXX only by TESS, and XXXX by both missions.


\subsection{Orbital periods}
Either Kepler/TESS or from RVs.
\subsection{Transit mid-times}
Use transit mid-times from the same mission if possible.
\subsection{Semi-major axes}
\label{sec:data:a}
The critical parameter is the distance between the star and the planet, not the semi-major axis itself. This varies if the orbit is eccentric. To account for varying distance, we adopt the mean semi-major axes from the literature, but use a custom estimate for the uncertainty that includes the eccentricity.
If the eccentricity is known, then the range of the distance is set to either the error on the semi-major axis or half of the difference between the semi-major and semi-minor axis, as calculated from the eccentricity -- whichever is larger. If eccentricity is NOT known, then the range of the distance is either the 25\% error on the semi-major axis, that is, assuming e=0.5 or the uncertainty on the semi-major axis -- whichever is larger.

\subsection{Rotation periods}
We adopt available rotation period values from the literature. When uncertainties are not given, we assume 10\% uncertainty.% _09_

\subsection{Planetary radii}
NASA table, see references in table, or using the planetary mass $M_p$ or $M_p\sin i$ to calculate the radius or a lower limit for the radius using the empirical relations derived in \cite{chen2017probabilistic} using their open source \texttt{forecaster} tool, upgraded to \texttt{astro-forecaster}\footnote{https://pypi.org/project/astro-forecaster/} by Ben Cassesse. 

We note that for GJ 674, the mass estimates in \cite{bonfils2007harps} and \cite{boisse2011disentangling} do not quote uncertainties, but differ by 0.3MEarth, so we assumed that value as the uncertainty on $M_p\sin i$.

\subsection{Bolometric luminosity}

We take bolometric luminosity values as given in the NASA Table if given with uncertainties, and supplement missing values and entries without quoted uncertainties with Gaia DR3 FLAME~\citep{fouesneau2022gaia} solutions. 

\section{Methods}
\label{sec:methods}
We measure flaring star-planet interactions (SPI) as the presence of excess flares triggered by the orbiting planet. In the absence of flaring SPI, flare peak times will be distributed randomly in orbital phase. In the presence of flaring SPI, we measure a phase dependent deviation from this randomness.

The main data for this analysis are flare times. To obtain them, we gather the Kepler and TESS light curves for all star-planet systems, remove rotational variability trends, search the de-trended light curves for flares, and convert the flare times to orbital phases. We then perform a customized Anderson-Darling test on the flare peak time distribution, which yields a p-value for the significance of the SPI signal. Finally, we compare these results to the theoretically expected SPI power $P_{SPI}$ in each system.

The methods for light curve de-trending and flare finding in Kepler and TESS light curves, as well as the Anderson-Darling test, are the same as detailed in~\citet{ilin2022searching}. We briefly recap the techniques in Sections~\ref{sec:methods:flaresearch} and \ref{sec:methods:adtest}. The expected power of SPI depends on the relative velocity between the planet and the magnetic field strength in its orbit, the derivation of which we explain in 
Sections~\ref{sec:methods:relvel} and \ref{sec:methods:bfield}, respectively. With relative velocity and magnetic field strength derived, we can combine them with stellar radius, planetary radius, and semi-major axis to estimate the power of SPI. We use the scaling laws from two different magnetic SPI models, which we introduce in Section~\ref{sec:methods:pspi}.

\subsection{Light curve de-trending and flare search}
\label{sec:methods:flaresearch}
To remove trends and rotational variability from the light curve without losing the flare signal, we use an empirically derived multistep process. First, we apply spline fit with a coarse sampling of 30h mean values to capture slow rotation with periods above multiple days and non-periodic trends. Then, we iteratively fit a series of sines to capture rotational signal on time scales down to about half a day, close to the fastest rotational signals measured in low mass stars. Eventually, we apply two Savitzky-Golay~\cite{savitzky1964smoothing} filters in sequence, with window sizes of 6h and 3h each. At this stage, we mask all data points above a $2.5 \sigma$ (or $1.5 \sigma$ for very active stars like AU Mic or Proxima Cen) threshold as flare candidates to prevent the filter from ironing out the flares. As a final step, we fit exponential functions to the edges of the light curves, if there are data points that deviate more than one standard deviation from the median value, while keeping the flare candidates masked.

In these light curves, we then search for flares as series of at least three consecutive data points $3\sigma$ above the noise level, implemented in AltaiPony~\cite{ilin2021altaipony}. We estimate the noise
level as the standard deviation in a rolling window of two hours, while masking deviation above $2.5\sigma$ (or $1.5$ sigma for active stars). To capture the exponential tail of the flare, we use the \texttt{addtail} flag to continue adding data points to the end of the flare as long as they are $2\sigma$ above the noise threshold. For these flares, we calculate the relative amplitude $a$ and equivalent duration $ED$, defined as the integrated flare flux $F_{flare}$ over the duration of the flare, divided by the median flux $F_0$ of the star, integrated over the flare duration~\citep{gershberg1972results}:
\begin{equation}
\label{eq:ED}
ED=\displaystyle \int \mathrm dt\, \frac{F_{flare}(t)}{F_0}.
\end{equation}
Equivalent duration is the time during which the non-flaring star releases as much energy as the flare.


\subsection{Custom Anderson-Darling test}
\label{sec:methods:adtest}
For each star-planet system with three or more flares in their Kepler and/or TESS light curves, we test for deviations from a random distribution of flares with orbital phase. We use the same customized Anderson-Darling test as in \cite{ilin2022searching}. In brief, we first take the number of flares observed in the TESS and Kepler data, calculate a base flare rate for each light curve. We then calculate how often each orbital phase has been covered by the data, again for each light curve. With the phase coverage and base flare rate combined, we can tell how many flares we would expect to see in any given phase bin if flares were randomly distributed. This number of expected flares per bin can be aggregated into one expected distribution by summing over all available light curves. As a last step, this distribution is then compared to the observed one using an Anderson-Darling test. We repeat this test with different phase offsets to account for potential biases in sensitivity of the test at different phases.

The only adjustment to the procedure in \cite{ilin2022searching} is that we use four equidistant start phases (i.e., 0, 0.25, 0.5 and 0.75) compared to the 20 used in~\cite{ilin2022searching} to save time, and because the range of derived $p$-values is already well-sampled using four. We adopt the standard deviation of these four $p$-values as the uncertainty on the flaring SPI measurement.


% \subsection{Orbital phase}
% \label{methods:orbitalphase}

% For the star-planet system that have transiting innermost planets, we use the transit mid-time to signify orbital phase zero. For non-transiting planets, we use arbitrary phase zero. The inferior conjunction time from the orbital parameters of planets that only have radial velocity measurements is usually either very uncertain, of the order of $10\%$ of the orbital duration~(XXXX), or cannot be derived because the eccentricity is poorly constrained~(XXXX). 


\subsection{Relative velocity}
\label{sec:methods:relvel}
We calculate the relative velocity between the magnetic field of the star, which we assume is co-rotating with the stellar rotation period $P_{rot}$, at the planetary orbit, where the planet moves with a period $P_{orb}$:

\begin{equation}
    v_{rel} = 2 \pi a \left(\frac{1}{P_{orb}} - \frac{1}{P_{rot}}\right).
\end{equation}

We use quadratic error propagation to estimate the uncertainty in $v_{rel}$ with the mean of the upper and lower uncertainty values on the orbital period and rotation period, if both are given, and the uncertainty on semi-major axis $a$ as derived in Sec.~\ref{sec:data:a}. 

\subsection{Stellar magnetic fields}
\label{sec:methods:bfield}
We derive the average magnetic field strength $B$ from $Ro$ using the empirical relation derived in \cite{reiners2022magnetism},  Table 2, in the unsaturated and saturated regimes, respectively:

\begin{eqnarray}
    B &= 199\,\text{G} \cdot Ro^{-1.26\pm 0.1} \;(\text{if}\; Ro > 0.13) \\
    B &= 2050\,\text{G} \cdot Ro^{-0.11\pm 0.03} \;(\text{if}\; Ro < 0.13) 
\end{eqnarray}

The estimate is consistent with existing Zeeman broadening for, e.g. AU Mic (3010 G~\cite{reiners2022magnetism})
\subsection{Power of star-planet interaction}
\label{sec:methods:pspi}
We consider two theories for the mechanism behind flaring star-planet interactions, the stretch-and-break and the Alfv\'en wing mechanisms. For both of them, we include the case of a magnetized and an unmagnetized planet, for a total of four estimates of the expected power of magnetic star-planet interaction.

First, we estimate the power of star-planet interaction generated by the stretch-and-break mechanism using the scaling relations in~\cite{lanza2012starplanet} (their Eqn. 45), where $P_{sb}$ is proportional to

\begin{equation}
    P_{sb} \sim B_p^{2/3} B_*^{4/3} v_{rel} R_p^2 a^{-4} R_*^4.
\end{equation}

$B_p$ and $B_*$ are the total planetary and stellar surface field strengths, respectively; $v_{rel}$ is the relative velocity between the stellar rotation and the planet's orbit at the semi-major axis; and $R_*$ and $R_p$ are the stellar and planetary radii, respectively; and $a$ is the semi-major axis.

Second, we also consider the case of an unmagnetized planet with $B_p=0$, in which case Eq.~45 in~\cite{lanza2012starplanet} reduces to

\begin{equation}
    P_{sb0} \sim B_*^2 v_{rel}  R_p^2  a^{-4} R_*^4
\end{equation}

Third, for comparison, we estimate the power of star-planet interaction generated by the Alfv\'en wing mechanism ~\cite{saur2013magnetic,kavanagh2022radio}, combining Eqs.~8~and~11 in \cite{kavanagh2022radio} to give

\begin{equation}
    P_{aw} \sim R_{p}^2  B_p^{2/3}  B^{1/3}  v_{rel}^2 a^{-2}  
\end{equation}
where we approximate the stellar wind density $\rho_W$ with $a^{-2}$ and stellar wind magnetic field at the planetary radius $B_W$ with the surface field strength $B a^{-3}$, which result in the total $a^{-2}$ factor.

And fourth, we finally consider the case of an unmagnetized planet with $B_p=0$, in which case Eq.~8 in \cite{kavanagh2022radio} reduces to

\begin{equation}
    P_{aw0} \sim R_p^2  B  v_{rel}^2  a^{-4}
\end{equation}


\section{Results}
\label{sec:results}

Our goal was to measure magnetic star-planet interaction as the presence of excess flares in phase with the innermost planet's orbit. We searched all star-planet systems that were observed with Kepler and TESS at 1 or 2 min cadence, respectively, for flares. For the star-planet systems in the resulting flare catalog~(Section~\ref{sec:results:catalog}), we confirmed that the orbital phase could be known well enough for the entire observing baseline of the system~(Section~\ref{sec:results:coherence}). Knowing that, we could move on to calculate the orbital phases of each flare, and calculate distributions for each system in the catalog~(Section \ref{sec:results:phasedist}). We then applied the custom Anderson-Darling test to check how likely it is that some of the flares were triggered by the planet. Comparing the test results with the expected power of magnetic star-planet interaction, we found that the probability of observing planet induced flares increases with the expected power in all considered scenarios~(Section~\ref{sec:results:spi}). However, not all systems with high expected power show signs of flaring SPI, which appears as a branching in the distribution of test results, and could be due to the intermittency of the interaction. Finally, we provide additional context for the most interesting systems in our sample, and explain why we had to exclude some systems from the analysis in Section~\ref{sec:results:individualstars}.
\subsection{Flare catalog}
\label{sec:results:catalog}
We searched a total of 3032 Kepler Quarters and 4181 TESS Sectors of a total of 1811 star-planet systems for flares. We inspected all candidates manually, and added flares observed by K2 on TRAPPIST-1~\cite{paudel2018k2}, and flares from the planet hosting primary in the Kepler-411 binary~\cite{jackman2021stellara}. The final table contains a total of \input{output/PAPER_total_number_of_flares.txt} flares in \input{output/PAPER_total_number_of_systems_with_flares.txt} systems. In Table~\ref{tab:flares}, we list all the flares and their characteristics. To identify each event, we provide the name of the system in which the flare was found, along with its TIC, whether it was observed with Kepler or TESS, and during what Quarter or Sector, respectively. For each flare, we also give the start and finish time, relative amplitude $a$ and equivalent duration $ED$. If the transit midtime is known for the innermost planet, we also give the orbital phase at which the flare occurred. The full table is provided with the supplementary material and via Zenodo (see Data Availability).

\begin{table*}
    \script{paper_latex_flare_table.py}
    \centering
            \caption{
            Flare catalog of all star-planet systems observed by Kepler and TESS. In transiting multi-planet systems, the orbital phase refers to the innermost planet, with the transit mid-time at phase zero. 
        }
    \input{output/flare_table.tex}
        \label{tab:flares}
\end{table*}

\subsection{Period coherence times}
\label{sec:results:coherence}
In many cases, the observing baseline covered by Kepler and TESS for a given system can span multiple years. In these cases, the orbital period of the innermost planet must be known very precisely so that we can assign accurate orbital phases to the flare events.
We test this by dividing the timespan between the first and the last flare in the combined Kepler and TESS observations by the coherence time $\tau$ of the orbital period for each system. We calculate the coherence time as

\begin{equation}
    \tau = P^2 / \sigma_P,
\end{equation}
where $P$ is the orbital period, and $\sigma_P$ is its uncertainty.
The resulting ratio should be $<<1$. The bottom panel in Figure~\ref{fig:coherence_hist} shows a histogram of these ratios in our sample. The orbital period is known sufficiently well for our analysis in most cases with ratios $<0.02$. The three cases where the ratio is highest with $0.05$, $0.12$, $0.24$ deviation are TAP 26, GJ 3082, and GJ 674, respectively. We keep these systems in our sample, but caution that the measured absence of flaring SPI might be due to the uncertain orbital period of the planet in these cases.

We repeated the coherence calculation for $P_{rot}$, and found that the ratio is $>.1$ for most stars~(Figure~\ref{fig:coherence_hist}, top panel). There are three reasons for this: First, the uncertainty for the rotation was not always provided. Second, rotation period is more difficult to ascertain from a light curves' variability than the orbital period, particularly if the modulation is weak and or rapidly evolving. And third, differential rotation causes spots at different latitudes to move at different speeds, so it becomes difficult to even define a rotation period, as long as we don't know at which latitude(s) the flares occur.

\begin{figure}[ht!]
    \script{paper_coherence_histogram.py}
    \begin{centering}
        \includegraphics[width=\linewidth]{figures/PAPER_coherence_histogram.png}
        \caption{
           Time span of observation vs. coherence time of the orbital and rotational periods, respectively. Orbital periods are all known precisely enough, so that the phase uncertainty at the last observed flare is of the order of $10^{-2}$. In contrast, in stars with known rotation periods, the precision is often too low, so that the phase of the last flare is often undetermined (ratio on x-axis $\sim 1$).
        }
        \label{fig:coherence_hist}
    \end{centering}
\end{figure}

\subsection{Flare phase distributions}
\label{sec:results:phasedist}
The Anderson-Darling test we use compares the measured distribution of orbital phases of the flares to the expected distribution, and return the significance of the deviation between the two. In the expected distribution, the same number of flares would be distributed randomly across the observing time. Figures~\ref{fig:cumdist_transiting} and \ref{fig:cumdist_rv} show the flare phase distributions along with the corresponding expected distributions for transiting and non-transiting systems, respectively. The expected distributions usually deviate from a straight line because we take into account the coverage of the orbital phases by the Kepler and TESS observations, as well as the different detection thresholds in each light curve.

\begin{figure*}[ht!]
    \script{paper_cumdist_individual_transiting.py}
    \begin{centering}
        \includegraphics[width=.95\linewidth]{figures/PAPER_flares_phase_hist_transiting.png}
        \caption{
            Cumulative distributions of orbital phases of flares in the transiting planet hosts observed by Kepler and TESS. The bisector line is dotted, the expected distribution is solid blue, and the observed distribution is solid black. Phase zero corresponds to the transit mid-time of the planet. 
        }
        \label{fig:cumdist_transiting}
    \end{centering}
\end{figure*}

\begin{figure*}[ht!]
    \script{paper_cumdist_individual_rv.py}
    \begin{centering}
        \includegraphics[width=0.95\linewidth]{figures/PAPER_flares_phase_hist_rv.png}
        \caption{
            Cumulative distributions of orbital phases of flares in the RV planet hosts observed by Kepler and TESS. The bisector line is dotted, the expected distribution is solid blue, and the observed distribution is solid black. Phase zero is chosen arbitrarily. 
        }
        \label{fig:cumdist_rv}
    \end{centering}
\end{figure*}

\subsection{Flaring star-planet interaction signal}
\label{sec:results:spi}
We used the custom Anderson-Darling test introduced in  \citet{ilin2022searching} and Section~\ref{sec:methods:adtest} to look for flaring SPI in all systems with three  or more flares with equivalent duration $ED>1\,$s, detected in their Kepler and TESS observations. Table~\ref{tab:maintable_der} lists the $p$-value results for each star-planet system. We do not find any $>3\sigma$ signal of star-planet interaction in our sample. However, overall, the significance of the deviation increases with higher expected power of interaction. 

Figure~\ref{fig:adtest_bp} shows that in every scenario we considered (Alfv\'en wing or stretch-and-break mechanism, with or without planetary magnetic field), there are two branches. On one branch, the significance of the measured interaction is correlated with the expected power $P$. On the other branch, no interaction is measured regardless of the expected power $P$. 

Table~\ref{tab:maintable_der} lists the values for the four different powers $P$, calculated with the scaling laws introduced in Section~\ref{sec:methods:pspi} and normalized to AU Mic, along with the derived parameters for these scaling laws, i.e., Rossby number R$_o$, surface-average magnetic field strength $B$, and relative velocity between the planet and a co-rotating magnetic field $v_{rel}$. 


\begin{table*}
\movetableright=-20mm
\footnotesize
    \script{paper_main_table.py}
    \caption{Flaring single star-planet system parameters.}
    \input{output/table_lit_vals.tex}
        \label{tab:maintable_lit}
    \tablerefs{\input{output/lit_table_bibstring.tex}}
\end{table*}


\begin{table*}
\footnotesize
\movetableright=-20mm
\script{paper_main_table.py}
\caption{Flaring star-planet interaction. $Ro$, $B$, and $v_{rel}$, are derived from literature values (Table~\ref{tab:maintable_lit}). $P_{xx}$ stands for the power of stretch-and-break ($sb$) and ALfv\'en wing ($aw$) interaction mechanisms, assuming the planet has a magnetic field strength of 1 G. $P_{xx0}$ is the same, but assuming an unmagnetized planet. All powers are normalized to AU Mic. The $p$-value of the Anderson-Darling test is lower when the system shows more flares periodic with the planetary orbit.}
\input{output/table_der_vals.tex}
    \label{tab:maintable_der}

\end{table*}


\begin{figure*}[ht!]
    \script{paper_adtest_vs_value_scatterplots.py}
    \begin{centering}
        \includegraphics[width=\linewidth]{figures/PAPER_ADtest_bg.png}
        \caption{
            Expected power of SPI vs. AD test results, assuming four different scenarios, color-coded by stellar surface field strength.
        }
        \label{fig:adtest_bp}
    \end{centering}
\end{figure*}



\subsection{Individual stars}
\label{sec:results:individualstars}
The properties of star-planet systems in this work are diverse, including both very young and old systems (with slowly rotating host stars); systems with super-Earths, Neptunes, and Hot Jupiters; and spectral types covering the lower main sequence from early G to late M dwarfs. 

For our analysis, it is important to know that the observed flares indeed occurred on the planet host star. We therefore excluded some systems due to documented contamination from nearby (bound or co-moving) companions~(Section~\ref{sec:results:individualstars:excluded}). One exception to this rule is Kepler-411, a binary system for which the flare contributions from each component could be separated~(Section~\ref{sec:results:individualstars:kep411}). We also dropped GJ 1061 from the sample because its rotation period was unconstrained. Another uncertainty in our analysis is the possibility of additional planets at even shorter orbits than the currently known innermost planet, such as might be the case for Proxima Cen~(Section~\ref{sec:results:individualstars:proxima}). 

The bulk of the systems in our sample are neither expected to show high power of magnetic star-planet interaction, nor do they show any deviation from intrinsic stellar flaring that could be indicative of flaring star-planet interaction. In the subsample of stars with high expected power, of the order of AU Mic's and above, we first take a closer look at those that seem to follow an increase in measured flaring star-planet interaction with increasing expected power $P$, that is AU Mic itself, K2-25, TOI-540, and HIP 67522~(Sections \ref{sec:results:individualstars:aumic}-\ref{sec:results:individualstars:hip67522}). Then we consider those systems where high power was expected in all scenarios, but not measured, i.e., TAP 26 and KOI-12~(Sections \ref{sec:results:individualstars:tap26} and \ref{sec:results:individualstars:koi12}).
\subsubsection{Excluded targets}
\label{sec:results:individualstars:excluded}

While they appear in Table~\ref{tab:flares}, we excluded a number of systems from further analysis. We drop all multiple stars except for Proxima Cen (whose components are well separated both physically and on the sky), and all stars with nearby objects that contaminate our analysis, e.g., from ~\citet{ziegler2018measuring}: \input{output/multiples_string.tex}

For instance, Kepler-808 (KOI 1300) has a $\sim0.4M_\odot$ \citep{kraus2016impact} companion that is about 1.8 mag fainter than the primary at a separation of .78 arcsec~\citep{baranec2016roboao}. TOI-837 has co-moving M dwarf a few arcsec away in the same young cluster, 5 mag fainter, but the primary is an F9-G0 star, so flares could still originate come from both stars~\citep{bouma2020cluster}. HD 41004 B, DS Tuc A, and LTT 1445 are known multiple systems where all components contribute significantly to the total flux.

The only single star system excluded is GJ 1061. The star is lacking a stellar rotation period estimate, which we need to calculate the magnetic field, and $v_{rel}$. \citet{dreizler2020reddots} estimate that it is a fairly slow rotator with a period between 50 and 200 days, so we do not expect it to have a strong magnetic field anyway.

\subsubsection{Kepler-411}
\label{sec:results:individualstars:kep411}
Kepler-411 has 3 mag fainter companion~\cite{wang2014influence,ziegler2018measuring}, but we do not exclude it from the analysis. The system was observed in Kepler short cadence by \citet{jackman2021stellara} and both the planet host, and the companion star flare. \citet{jackman2021stellara} disentangle the contributions from each component on a pixel level. According to \citet{morton2016false, sun2019kepler411}, Kepler-411's planets orbit the primary companion. The fainter companion of Kepler-411 appears to cause the majority of flares (41), whereas the planet host causes only 7. We adopt the 7 flares from \cite{jackman2021stellara} in our analysis.

\subsubsection{Proxima Cen}
\label{sec:results:individualstars:proxima}
We do not detect a deviation from intrinsic flaring on Proxima Cen, consistent with numerical models that place Proxima Cen b well outside the sub-Alfv\'enic zone~\citep{kavanagh2021planetinduced}. However, this might not entirely exclude Proxima Cen from the search for flaring star-planet interaction. The tentatively detected Proxima Cen d, a planet further in, at 0.029 au, or 5 day orbital period~\citep{faria2022candidate, artigau2022linebyline} could still cause flaring star-planet interactions. However, the orbital period of the tentative planet is so uncertain at this point that its coherence time is shorter than the roughly 2yr observing baseline of Proxima Cen, preventing us from accurately measuring the flare phases. 


\subsubsection{AU Mic}
\label{sec:results:individualstars:aumic}

AU Mic is a 16-29 Myr old pre-main sequence M0-M1 dwarf with a strong magnetic field of about $3010\pm220\,$G obtained from Zeeman broadening measurements~\citep{reiners2022magnetism}. The innermost Neptune-sized planet, AU Mic b, could be both inside the sub-Alfv\'enic zone if the star's mass loss rate is relatively low at about 30 times the solar value or lower. If the mass loss rate is high, exceeding the solar value by a factor of several hundreds, AU Mic b could orbit in the super-Alfv\'enic zone, and become unable to experience any planet-induced flaring. Recently, \citet{klein2022one} found tentative periodicity with the orbital period of AU Mic b in the chromospheric He I D line, which increases if the contribution from flares is included in the calculation. Overall, our marginal signal of flaring star-planet interaction in the AU Mic system supports the idea that this young, magnetically active M dwarf system with a close-in Neptune could exhibit planet=induced flaring.
For a detailed discussion of AU Mic and its flaring star-planet interaction, we refer to~\cite{ilin2022searching}, where we also estimate that an additional 50–100 days of TESS-like monitoring of AU Mic would yield a $3\sigma$ detection if the marginal signal in our data is real.
 
\subsubsection{K2-25}
\label{sec:results:individualstars:k225}
\subsubsection{TOI-540}
\label{sec:results:individualstars:toi540}

\subsubsection{HIP 67522}
\label{sec:results:individualstars:hip67522}
HIP 67522 has one of the strongest expected SPI signals in our sample, and shows the clearest signs of flaring SPI, albeit with only 6 flares in the sample. These flares are distributed across two Sectors in TESS. In Sector 11, two flares occur at orbital phases $\sim 0.6$ and $\sim 0.8$. Two years later, in Sector 38, four flares occur, but this time, all of them take place at orbital phases $0.00-0.05$, that is, up to a few hours after transit.

HIP 67522 is young Sun, currently contracting onto the main sequence. It is a Sco-Cen member (10-20 Myr old), which was discovered to host a close-in Jupiter, HIP 67522 b, in 2020~\citep{rizzuto2020tess}. It was found with low obliquity, consistent with XXXX~\citep{heitzmann2021obliquity}. Tentative third planet transiting.

HIP 67522 is in close spin-orbit commensurability -- $P_{rot}/P_{orb}=5/1$. The system's $a/R_*\approx11.7$, eccentricity below .25~\cite{rizzuto2020tess}, so that  

\cite{wood2021characterizing} rule out stellar companions for HIP 67522 based on RV, high-resolution imaging, and Gaia data.

HIP 67522 will be observed again with TESS in spring 2023.

\subsubsection{TAP 26}
\label{sec:results:individualstars:tap26}

\subsubsection{KOI-12}
\label{sec:results:individualstars:koi12}






%\subsection{Combined Sample}

%For those stars with given transit mid-time and 

\section{Discussion}
\label{sec:discussion}

\subsection{Similar work that puts ours in context}

Any indication of flaring SPI anywhere else? Any reason to think there should be any?
Why does HIP 67522 stick out so much? -- spin-orbit alignment

\subsection{Rotational variability}
Hard to measure, because it de-phases much faster than orbital variability due to differential rotation and spot evolution. We could use single light curves within which coherence is given...

\subsection{Intermittency}
\label{sec:discussion:intermittency}
TAP 26 vs. HIP 67522, also different sectors for HIP 67522, see \cite{shkolnik2008nature}

\subsection{Spin-orbit alignment}
nother relevant factor is the geometry of the system. If we assume that the magnetic dipole axis and the rotation axis of the star are relatively well-aligned, the spin-orbit alignment may have an influence on interaction strength. 


or more broadly system architecture including misalignment of the magnetic axis. See HAT-P-11, or instance, and the many polar orbits in \cite{bourrier2023dream}

\cite{miskovetz2022resolving}  found background stars that were not co-moving HAT-P-11. HAT-P-11 rotates nearly pole-on, and HAT-P-11 b is transiting in a near-polar orbit~\citep{bourrier2023dream}. This may be the reason for absent SPI.
\subsection{Eccentricity}
The star-planet systems in this work do not have notably eccentric orbits.

\section{Summary and Conclusions}

\bibliography{bib}

\appendix
\section{Notes on individual stars}
\subsection{Proxima Cen}
Although Proxima Cen is part of a triple stellar system, we treat it a single star in this work because of the large angular separation to its two companions. Proxima Cen b has an eccentricity upper limit of 0.35~\cite{anglada-escude2016terrestrial}.
\subsection{TAP 26}
Likely low eccentricity given in \cite{yu2017hot}.

\subsection{GJ 393}
Eccentricity likely low, but not quantified in \cite{amado2021carmenes}, so we treat it as unconstrained (set e=0.5)

\subsection{Kepler-42}
 KOI-961 in the detection paper \cite{muirhead2012characterizing}, low eccentricity constrained by \cite{mann2017gold}


\subsection{K2-354}
We note that we found K2-354 under EPIC 211314705 and K2-329(!) in \cite{bouma2020cluster}, who refer to the detection paper \cite{pope2016transiting}, and also under TIC 468989066.
\end{document}



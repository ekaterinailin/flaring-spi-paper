% Define document class
\documentclass[twocolumn]{aastex631}
\usepackage{showyourwork}

% Begin!
\begin{document}

% Title
\title{Flaring star-planet interactions in Kepler and TESS}

% Author list
\author{Ekaterina Ilin}
\author{Katja Poppenh\"ager}
\author{Judy Chebly}
\author{Nikoleta Ilic}
\author{Juli\'an Alvarado-G\'omez}



% Abstract with filler text
\begin{abstract}
    In many of the star-planet systems discovered so far, the innermost planets orbit within only a few stellar radii. In these systems, the planets become in-situ probes of the extended stellar magnetic field. As they move, they can trigger flares on the stellar surface. Until recently, however, searches for such planet-induced flaring suffered from small sample sizes, and poor orbital phase coverage.

    In this work, we used the excellent phase coverage of the archives of Kepler and the Transiting Exoplanet Survey Satellite to search for star-planet systems with flaring star-planet interactions. We compiled a catalog of all their flares, and searched for planet-induced flaring, i.e., flares clustering in phase with the planet's orbit. 

    In the 1811 searched systems, we found 25 single stars with three or more flares each. We quantified the significance of their excess flaring, and compared it against the theoretically expected amplitude of magnetic interaction. As we did not select for short orbits, most systems do not show excess flaring, as expected. Those we expect to interact fall on two branches. An inactive one, without any signs of excess flaring, and an active one, where expected power and excess flaring are correlated. This trend is tentative, but consistent for several scaling laws for magnetic interaction. 
    
    Finally, we searched our data for locally expressed tidal interaction, i.e., excess flaring in phase with the tidal bulges ($P_{\rm orb}/2$). We find a marginal trend with different scaling laws of tidal interaction. Our system-by-system comparison tentatively suggests that young systems can interact both tidally and magnetically, whereas old, magnetically inactive systems mostly interact tidally.
    %magnetic field strength (using $Ro$-$B$ relations), orbital parameters, and planet radius.
\end{abstract}

% Main body with filler text
\section{Introduction}
\label{sec:intro}
One of the most surprising results in exoplanet research is the diversity of star-planet system architectures. The past two decades revealed that planets can occur in orbits of only a few stellar radii~[e.g., ][]\citep{sanchis-ojeda2014study} that they could almost be considered to live inside the stellar atmosphere. At this proximity, the planet exerts force, tidal and magnetic, both short- and long-term, on its host -- a phenomenon largely unknown to the Solar System. These forces are typically subsumed under the term star-planet interactions (SPI) that emphasizes that they are bidirectional, affecting both planet, and star. Specifically, the effects on the star pose challenges, but also opportunities for our understanding of star-planet systems:

On the one hand, a close-in planet that changes the star's rotation and activity makes it more difficult for us to model the system's evolution, as it affects fundamental stellar parameters. Ages inferred from rotation will be biased, if the star is spun up or spun down relative to stars without close-in companions~\citep{tejadaarevalo2021further, brown2014discrepancies, maxted2015comparison}. Analogously, altered activity levels may mimic an older or younger host than it actually is~\citep{ilic2022tidal}. On the other hand, measuring in what ways and how much close-in planets alter their hosts' behavior can provide important insights into the habitability of the entire system~\citep{shkolnik2018signatures}. Even if they orbit far inside the habitable zone, such planets probe the system's space weather and mass loss by revealing the extent of the Alfv\'en zone through SPI~\citep{kavanagh2021planetinduced, chebly2022destination}. % Prospectively, planet-induced stellar activity could quantify otherwise difficult to measure planetary properties such as the strength and extent of their magnetic field~\citep{cauley2019magnetic}. 

The search for magnetic SPI (Section~\ref{sec:intro:mspi}) in the form of changes in global activity indicators~(Section~\ref{sec:intro:global}), has so far been a series of detections and non-detections, often due to selection bias in the studied systems. Looking for local changes that trace the magnetic footpoint that connects planet to star mitigates this bias, but introduces another -- sampling bias due to poor orbital phase coverage. Flares~(Section~\ref{sec:intro:flares}) are activity markers that can be caused by SPI~(Section~\ref{sec:intro:fspi}). They can be monitored with high phase coverage over long periods of time with missions like Kepler and TESS. Thanks to their large archives of hundreds of cumulative years of stellar monitoring, we can overcome the limitations of individual system studies~(Section~\ref{sec:intro:local}), and now look for flaring SPI more comprehensively, and systematically.

\subsection{Magnetic star-planet interactions}
\label{sec:intro:mspi}
Magnetic star-planet interactions can occur if the planet is orbiting within the Alfv\'en radius of the star at least part of the time~\citep{preusse2006magnetic, cohen2011dynamics}. The Alfv\'en radius is the radius at which the stellar wind velocity, which increases with increasing distance from the star, exceeds the Alfv\'en velocity of the magnetized plasma. Beyond this radius, this plasma is disconnected from the star. Alfv\'en waves set off by the planet within the magnetic field they cross cannot propagate back to the star if the wind carries the plasma away faster than these waves can travel. In the sub-Alfv\'enic regime, the wave can reach the star, and deposit its energy in its atmosphere. \citet{lanza2012starplanet, lanza2018closeby}, and \citet{zarka2007plasma} and \citet{saur2013magnetic} suggest different mechanisms for when and how the energy transport and dissipation takes place, and how much energy can be transferred and dissipated at all. They all lack observational constraints from large samples for calibration, which we aim to provide in this work.

\subsection{Searching for global changes in activity}
\label{sec:intro:global}
Measuring magnetic star planet-interaction is notoriously difficult in individual systems because a non-detection may not mean an absence of the interaction, but merely a temporary cessation~\citep{shkolnik2005hot, shkolnik2008nature}. Statistical studies searching for changes in global chromospheric and coronal activity indicators in star-planet systems with close-in planets relative to planet-less stars are inconclusive~\citep{kashyap2008extrasolar,scharf2010possible, shkolnik2013ultraviolet, france2018farultraviolet, viswanath2020statistical, krejcova2012evidence, miller2015comprehensive, poppenhaeger2010coronal}. In these studies, it is challenging both to define a control sample to measure the activity against, and to quantify how the way we detect planets introduces selection bias in the activity measure. It is, for instance, easier to detect a planet around a magnetically inactive star, because both radial velocity and transit detections will be less affected by stellar variability~\citep{poppenhaeger2011correlation}. Tidal interactions can also induce global increase in activity~\citep{ilic2022tidal}, which adds further ambiguity to the interpretation. 

\subsection{Flares}
\label{sec:intro:flares}
In this study, we will look at stellar flares as another way of dissipating the energy deposited in the magnetic field. Flares are strong and impulsive eruptions released by reconnection and subsequent relaxation of field lines in the corona~\citep{svestka1976solar,priest2002magnetic}. In contrast to solar flares, stellar flares often are extremely energetic, sometimes enhancing the stellar flux by orders of magnitude~\citep{maehara2012superflares, shibayama2013superflares, paudel2018k2} so that they can be detected in stars over 1kpc away~\citep{chang2015photometric}. But since they occur randomly in time, individual events can be hard to catch. %We can observe flares in nearly all electromagnetic bands -- from radio to X-ray~\citep{benz2010physical}.

In the 2010s, a wealth of flare observations from space missions like Kepler~\citep{borucki2010kepler} and K2~\citep{howell2014k2} have been catalogued~\citep{davenport2016kepler, paudel2018k2, ilin2021flares}. We now have access to long-term stellar monitoring data available for flare hunting. Since 2018, the Transiting Exoplanet Survey Satellite~\citep{ricker2015transiting} is rapidly growing the archive~\citep{gunther2020stellar}. The key advantage of both missions for this work is the nearly uninterrupted monitoring of star-planet systems that covers all phases of a close-in planet's orbit.

\subsection{Flaring star-planet interactions}
\label{sec:intro:fspi}
The absolute power of flaring star-planet interactions is difficult to constrain. We do not know how energetic planet-induced flares can be. The estimates for dissipated energy~\citep{lanza2018closeby} are in the ballpark of the regime of superflares observed by Kepler and TESS. However, on the one hand, the planet might be triggering flares prematurely, which would be of lower energy~\citep{loyd2023flares}, and end up below our detection threshold. On the other hand, the energy need not be dissipated continuously, particularly in the stretch-and-break mechanism proposed by~\citep{lanza2012starplanet}, so an individual flare's energy might be higher. We also do not know whether flares triggered by interactions with a companion would have different properties than intrinsic flares. We will assume that, since they would be caused by magnetic reconnection triggered by the planet in the stellar corona, they should still have the same physics as intrinsic flares. 

Since planet-induced flares might not look any different from intrinsic flares, and we do not know how their energies will be distributed, we require a method to distinguish them without referring to their individual properties. Instead, we can look at the statistical distribution of flares in time: 

If we picture the magnetic field lines connecting the planet to the star in the sub-Alfv\'enic zone, we expect that the stellar footpoint of these lines will move along with the planet. Flares triggered by the planet will occur at a preferred orbital phase, namely whenever the footpoint is visible to the observer. Depending on the system's viewing geometry, i.e., the inclination of the rotation axis, the spin-orbit (mis-)alignment, and the extent and configuration of the large scale magnetic field -- the footpoint could be always in view, or never. In expectation, however, in star-planet systems with favorable viewing geometries and magnetic field strengths and configurations, the footpoint visibility will be modulated in phase with the planetary orbit. The resulting phase preference of planet-induced flares introduces a deviation from a random distribution of flares. Space missions like Kepler and TESS provide light curves that cover all phases of the planet's orbit, so that phase-correlated planet-induced flares can be detected against a background of uncorrelated intrinsic flaring. 

\subsection{Searching for local changes in activity}
\label{sec:intro:local}
In contrast to global SPI, local SPI effects are easier to tell apart from intrinsic stellar behavior because they are tied to the planet's orbital period. 
Phase-correlated activity, flaring and otherwise, has been searched for in individual systems. After eight years of monitoring of HD 189733, an active Hot Jupiter host, higher X-ray energy flares seem only marginally clustered in orbital phase than lower energy ones~\citep{pillitteri2022xray}. \citet{maggio2015coordinated} observed an eruption of X-ray emission when the Jupiter-sized companion was in periastron on the highly eccentric orbit around its host HD 17156. However, a similar study using chromospheric indicators in the eccentric system HD 80606 found no variation between periastron and apoastron~\citep{figueira2016activity}. In the above cases, non-detection can at least partly be attributed to poor orbital phase coverage, which also limits the significance of any positive detection. Realizing this, \citet{fischer2019timevariable} used K2 data with excellent phase coverage of the orbits of the inner \mbox{TRAPPIST-1} planets to look for clustered flaring in phase, but found none. \citet{ilin2022searching} and \citet{klein2022one} found hints of excess flaring, and chromospheric Helium emission in the young AU Mic system, respectively. This mixed picture of individual object studies calls for a more comprehensive and systematic approach.

\subsection{Overview}
In this work, we searched for excess flaring in all available Kepler and TESS short cadence light curves of confirmed star-planet systems~(Section~\ref{sec:data}). We scoured the light curves for flares, and tested their observed phases for departures from a uniform distribution in phase with the innermost planet's orbit~(Section~\ref{sec:methods}). We present the catalog of all flares found in our sample of star-planet systems, and compare the test results with the expected power of interaction for each system~(Section~\ref{sec:results}). We discuss the results for the most interesting star-planet systems individually~(Section~\ref{sec:results:individualstars}), and take a look at the patterns emerging from our analysis~(Section~\ref{sec:discussion}), and the possible role of tidal flaring SPI in our observations~(Section~\ref{sec:discussion:tidal}).  We summarize our findings in Section~\ref{sec:summary}. 




%Short intro to SaB mechanism

%Short intro to AW mechanism

% ----------------------------------------------------------------------------------

% In both models, the radius of the perturbing body, and its proximity increase the power of interaction. Would it not be more reasonable to search for star-brown dwarf or star-star interactions? 

% Flares caused by magnetic interactions in close binary systems were previously observed. Flares detected on the young eccentric binary system V 773 Tau could be attributed to colliding magnetospheres~\citep{massi2002periodic, massi2008interacting}, while other promising systems, like DQ Tau, another young equal-mass binary system in a strongly eccentric orbit, did not live up to the expectations~\citep{salter2008captured,salter2010recurring,getman2011young,getman2016search, kospal2018spots}. In these systems, it is, however, unclear whether it is mass transfer or reconnection that drive excess coronal emission and flares at periastron. Substellar objects can orbit the star much closer than a stellar companion without filling the Roche lobe. Gravitationally, a rocky planet can go unnoticed by its host even if it orbits at ultrashort periods of less than a day. 
% Interactions in systems with brown dwarf companions are less likely than binary systems 



\section{Data}
\label{sec:data}
In this study, we aimed at compiling the largest possible sample to search for flaring star-planet interactions. We took the entire Planetary Systems Composite Parameters Table (PSCP\footnote{ \url{https://exoplanetarchive.ipac.caltech.edu/cgi-bin/TblView/nph-tblView?app=ExoTbls&config=PSCompPars}}, Section~\ref{sec:data:sps}) from the NASA Exoplanet Archive, as of July 2022, and scanned the systems for available Kepler and TESS light curves~(Section~\ref{sec:data:photometry}). We searched a total of 1811 systems and over 7200 light curves from both missions for flares. To calculate the amount of excess flaring from star-planet interaction, we further required the orbital periods of each system~(Section~\ref{sec:data:orbitalperiod}). To compare our measurements to theory, we estimated the expected power of interaction for each system with at least three flares. For this, we required further system parameters -- distance between planet and star, stellar rotation period, planetary radius, and stellar luminosity~(Sections~\ref{sec:data:a}-\ref{sec:data:lum}), which we mostly took from the literature. We began by compiling the table of system parameters with the PSCP, but verified, supplemented and updated the table with more up-to-date results. The parameters for the systems in the final sample of 25 systems are shown in Table~\ref{tab:maintable_lit}.


\begin{table*}
\movetableright=-20mm
\footnotesize
    \script{paper_main_table.py}
    \caption{Flaring single star-planet system parameters.}
    \input{output/table_lit_vals.tex}
        \label{tab:maintable_lit}
    \tablerefs{\input{output/lit_table_bibstring.tex}}
\end{table*}


\subsection{Star-planet systems}
\label{sec:data:sps}
We compiled our sample of star-planet systems from the PSCP, from which we removed controversial planet detections (``pl\_controv\_flag'' must be 0), i.e., detections with existing literature questioning the result. The resulting catalog contained 2993 transiting and 191 non-transiting systems, from which we picked the innermost known planet in each system. 

\subsection{Kepler and TESS photometry}
\label{sec:data:photometry}
The Kepler and TESS missions are unbeaten with respect to long-term optical monitoring of stellar flares. Their excellent coverage of orbital phases makes the light curves ideal for our search for orbital phase dependent flaring. Between 2009 and 2013, the Kepler space telescope~\citep{koch2010kepler} nearly continuously observed a patch of the sky in the Cygnus-Lyra region. Each of the 18 observing Quarters contains nearly uninterrupted $\sim 90$ days of observations, totaling over $100\,000$ stars monitored in 2-min cadence in a broad $400-850\,$nm bandpass.
% 100000 from here: https://exoplanetarchive.ipac.caltech.edu/docs/KeplerMission.html

The Transiting Exoplanet Survey Satellite~[TESS, ][]\citep{ricker2015transiting} is an all-sky mission that began operations in 2018, completed two sky scans until summer 2022. It is observing at the time of writing, collecting nearly continuous photometric time series in the $600-1000\,$nm band for $\sim 27\,$d in each observing Sector. About $200\,000$ stars have been observed in 2-min cadence in the first two years of operations alone, with about $20,000$ targets per Sector. Out of these, from Sector 27 on, $1,000$ targets were observed at even higher 20-s cadence in each Sector. 
% 20000 per sector does not add up, because of stars observed in multiple sectors, the 20000 and 1000 figures can be found here: https://tess.mit.edu/observations/target-lists/
% extended TESS mission https://heasarc.gsfc.nasa.gov/docs/tess/extended.html

Based on the filtered PSCP table, we use the \texttt{lightkurve}~\citep{lightkurvecollaboration2018lightkurve} Python software to query the full Kepler archive (quarters Q0-Q17, Data Release 25), and the most recent TESS catalog (July 2022) for their respective 1-min and 2-min/20-s cadence light curves. In total, we searched 7258 light curves for flares -- 3030 Kepler Quarters, and 4228 TESS Sectors. We found and analyzed light curves for a total of 1811 systems, 344 of which were observed only by the primary Kepler mission, 1205 only by TESS, and 262 by both missions. We did not use K2 light curves because of the many systematics, and consequently the high time investment of de-trending them. But we included the flare list obtained by~\citet{paudel2018k2} for TRAPPIST-1 K2 flares, since this system is a potential candidate for sub-Alfv\'enic interactions~\citep{fischer2019timevariable}.

\subsection{Orbital periods and transit mid-times}
\label{sec:data:orbitalperiod}
We adopt the orbital periods from the PSCP, which were either obtained from Kepler or TESS transits, or from radial velocity measurements. For the transiting systems, we use the transit mid-times given in the PSCP to set orbital phase zero. If possible, we use the transit mid-times determined in each mission separately to reduce the uncertainties on orbital phase. For the non-transiting system, phase zero is set arbitrarily.
\subsection{Semi-major axes and eccentricities}
\label{sec:data:a}
A critical parameter for the possibility of magnetic SPI and its power is the distance between the star and the planet, not the semi-major axis itself. This varies if the orbit is eccentric. We therefore adopt the mean semi-major axes from the literature, but use a custom estimate for the uncertainty that includes the eccentricity:

If the eccentricity is known, then the uncertainty on the distance is set to either the error on the semi-major axis, or to half of the difference between periastron and apoastron -- whichever is larger. If eccentricity is not known, then the uncertainty on the distance is set to either the 25\% error on the semi-major axis, that is, assuming $e=0.25$, or the uncertainty on the semi-major axis -- whichever is larger. We chose $e=0.25$ because it is both a typical value within our sample~(see Table~\ref{tab:maintable_lit}), and in the literature~\citep{eylen2019orbital}.

\subsection{Rotation periods}
\label{sec:data:rotationperiods}
We adopt available rotation period values from the literature. Almost all stem from light curve variability~\citep{angus2018inferring, mazeh2015photometric, mcquillan2013stellar, mcquillan2014rotation, luger2017sevenplanet, stock2020carmenes, deleon202137, torres2017validation, stefansson2020habitable, zicher2022one, ment2021toi, rizzuto2020tess, gunther2020stellar}, and only one from periodic variation in chromospheric lines~\citep{demangeon2021warm}. When uncertainties are not given, we conservatively assume $10\%$ uncertainty. Uncertainties are missing usually when periods were detected using Lomb-Scargle~\citep{lomb1976leastsquares, scargle1982studies}, or similar, periodogram peaks only~\citep{gunther2020stellar, kiraga2007agerotationactivity, grankin2013magnetically, burt2014lickcarnegie}, but also when the measurement is indirect using activity-rotation relations~(only in the case of GJ 3323).

Among the stars without given uncertainties, the rotation of GJ 3082 was measured and is consistent both in TESS and in KELT light curves to below 1\,s precision~\citep{gunther2020stellar}. The rotation of GJ 674 is consistent between light curve periodograms~\citep{kiraga2007agerotationactivity}, and activity-rotation relations~\citep{boisse2011disentangling} at a $<10\%$ level. GJ 3323 only has rotation periods inferred from activity indicators~\citep{astudillo-defru2017magnetic}, without any periodicity detected in the activity indicators themselves~\citep{astudillo-defru2017harps}. The rotation of Proxima Cen is very slow, but consistent between different datasets~\citep{anglada-escude2016terrestrial, kiraga2007agerotationactivity}. In general, among the stars without given uncertainties on the rotation period, the period is either 
\begin{itemize}
    \item short, leaving a clear peak in the periodogram, and little doubt about the period after ruling out aliases with $P/2$, as with GJ 3082, or
    \item longer than the Sun's, that is, in the regime where rotation-activity relations are reliable because the stars are in the unsaturated activity regime, as in the case of GJ 674, and GJ 3323.
\end{itemize} 


\subsection{Planetary radii}
\label{sec:data:planetradii}
For the transiting planets, we adopt the literature values for planet radius $R_p$, and uncertainties, from the PSCP. For the non-transiting radial velocity detected planets we use the planetary mass $M_p$ or $M_p\sin i$ to calculate the radius or a lower limit for the radius using the empirical relations derived in \cite{chen2017probabilistic} using their open source \texttt{forecaster} tool, upgraded to \texttt{astro-forecaster}\footnote{https://pypi.org/project/astro-forecaster/} by Ben Cassesse. We note that for GJ 674, the mass estimates in \cite{bonfils2007harps} and \cite{boisse2011disentangling} do not quote uncertainties, but differ by $0.3M_\oplus$, so we assumed that value as the uncertainty on $M_p\sin i$.

\subsection{Bolometric luminosity}
\label{sec:data:lum}

We take bolometric luminosity values as given in the PSCP whenever they are given with uncertainties. We supplement missing values, and replace entries without quoted uncertainties with Gaia DR3 FLAME~\citep{fouesneau2022gaia} solutions~(\texttt{lum\_flame, lum\_flame\_upper, lum\_flame\_lower}). 

\section{Methods}
\label{sec:methods}
We measure flaring star-planet interactions (SPI) as the presence of excess flares triggered by the orbiting planet. In the absence of flaring SPI, flare peak times will be distributed randomly in orbital phase. In the presence of flaring SPI, we measure a phase dependent deviation from this randomness.

The main data for this analysis are flare times. To obtain them, we gather the Kepler and TESS light curves for all star-planet systems, remove rotational variability trends, search the de-trended light curves for flares, and convert the flare times to orbital phases. We then perform a customized Anderson-Darling test on the flare peak time distribution, which yields a $p$-value for the significance of the SPI signal. Finally, we compare these results to the theoretically expected SPI power $P_{xx}$ in each system.

The methods for light curve de-trending and flare finding in Kepler and TESS light curves, as well as the Anderson-Darling test, are the same as detailed in~\citet{ilin2022searching}. We briefly recap the techniques in Sections~\ref{sec:methods:flaresearch} and \ref{sec:methods:adtest}. The expected power of SPI depends on the relative velocity between the planet and the magnetic field strength in its orbit, the derivation of which we explain in 
Sections~\ref{sec:methods:relvel} and \ref{sec:methods:bfield}, respectively. We can then combine them with stellar radius, planetary radius, and semi-major axis to estimate the power of SPI. We use the scaling laws for two different magnetic SPI models, which we introduce in Section~\ref{sec:methods:pspi}.

\subsection{Light curve de-trending and flare search}
\label{sec:methods:flaresearch}
To remove trends and rotational variability from the light curve without losing the flare signal, we use an empirically derived multistep process, implemented as the \texttt{custom\_detrending} method in \texttt{AltaiPony}, a flare science Python package for light curve analysis~\citep{ilin2021altaipony}. First, we apply a spline fit with a coarse sampling of 30h-averaged values to capture slow rotation with periods above multiple days and non-periodic trends. Then, we iteratively fit a series of sines to capture rotational signal on time scales down to about half a day, close to the fastest rotational signals measured in low mass stars. Eventually, we apply two Savitzky-Golay~\citep{savitzky1964smoothing} filters in sequence, with window sizes of 6h and 3h each. At this stage, we mask all data points above a $2.5 \sigma$ (or $1.5 \sigma$ for stars like AU Mic or Proxima Cen, which are very active or have very low noise levels, or both) threshold as flare candidates to prevent the filter from ironing out the flares. As a final step, we fit exponential functions to the edges of the light curves, if there are data points that deviate more than one standard deviation from the median value, while keeping the flare candidates masked. Transits are usually too shallow to affect either light curve de-trending or flare finding. The light curve portions around deep transits were inspected manually.

In the de-trended light curves, we search for flares as series of at least three consecutive data points $3\sigma$ above the noise level using the \texttt{AltaiPony} method \texttt{find\_flares}. We estimate the noise level as the standard deviation in a rolling window of two hours, while masking deviation above $2.5\sigma$ (or $1.5\sigma$  for stars like AU Mic and Proxima Cen). To capture the exponential tail of the flare, we use the \texttt{addtail} flag in the \texttt{find\_flares} method to continue adding data points to the end of the flare as long as they are $2\sigma$ above the noise threshold.

We vet all flare candidates by eye, and exclude instrumental false positives and physical ones like passing Solar System Objects. For each confirmed flare candidate, we calculate the relative amplitude $a$ and equivalent duration $ED$, defined as the integrated flare flux $F_{\rm flare}$ over the duration of the flare, divided by the median flux $F_0$ of the star, integrated over the flare duration~\citep{gershberg1972results}:
\begin{equation}
\label{eq:ED}
ED=\displaystyle \int \mathrm dt\, \frac{F_{\rm flare}(t)}{F_0}.
\end{equation}
Equivalent duration is equal to the time during which the non-flaring star releases as much energy as the flare.


\subsection{Custom Anderson-Darling test}
\label{sec:methods:adtest}
For each star-planet system with three or more flares in its light curves, we test for deviations from a random distribution of flares with orbital phase. We use the same customized Anderson-Darling test as in \cite{ilin2022searching}. In brief, we first take the number of flares observed in the TESS and Kepler data to calculate a base flare rate for each light curve. We then calculate how often each orbital phase has been observed, also for each light curve separately. With the phase coverage and base flare rate combined, we can tell how many flares we would expect to see in any given phase bin if flares were randomly distributed. We then aggregate this number of expected flares per time bin into one expected distribution by summing over all available light curves. As a last step, we compare the expected distribution to the observed one using an Anderson-Darling test. We repeat this test with different phase offsets to account for potential biases in sensitivity of the test at different phases.

The only adjustment to the procedure in \cite{ilin2022searching} is that we use four equidistant start phases (i.e., 0, 0.25, 0.5 and 0.75) compared to the 20 used in~\cite{ilin2022searching} to save time, because the range of derived $p$-values is already well-sampled using four. We adopt the standard deviation of these four $p$-values as the uncertainty on the flaring SPI measurement.



\subsection{Relative velocity}
\label{sec:methods:relvel}
We calculate the relative velocity between the magnetic field of the star and the planet using stellar rotation period $P_{\rm rot}$, orbital period $P_{\rm orb}$, and semi-major axis $a$. We assume that the large scale magnetic field is co-rotating with the stellar surface, and calculate the relative velocity at the planet's orbital distance:

\begin{equation}
    v_{\rm rel} = 2 \pi a \left(\frac{1}{P_{\rm orb}} - \frac{1}{P_{\rm rot}}\right).
\end{equation}

We use quadratic error propagation to estimate the uncertainty in $v_{\rm rel}$ with the mean of the upper and lower uncertainty values on the orbital period and rotation period, if both are given, and the uncertainty on semi-major axis $a$ as derived in Sec.~\ref{sec:data:a}. 

\subsection{Stellar magnetic fields}
\label{sec:methods:bfield}
We derive the average magnetic field strength $B$ from $R$o using the empirical relation derived in \citet{reiners2022magnetism},  Table 2, in the unsaturated and saturated regimes, respectively:

\begin{eqnarray}
    B &= 199\,\text{G} \cdot R\rm o^{-1.26\pm 0.1} \;(\text{if}\; Ro > 0.13) \\
    B &= 2050\,\text{G} \cdot R\rm o^{-0.11\pm 0.03} \;(\text{if}\; Ro < 0.13) 
\end{eqnarray}

The convective turnover time $\tau$ in  $R \rm o=P_{rot}/\tau$ is derived using bolometric luminosity~(Section~\ref{sec:data:lum}), following~\citet{reiners2014generalized, reiners2022magnetism}:

\begin{equation}
    \tau = 12.3\, \rm d \cdot (L_{\rm bol}/L_\odot)^{-1/2}
\end{equation}

We compared our derived values for $R$o and $B$ with existing estimates of coronal emission from~\citet{foster2022exoplanet}. We show $L_X/L_{\rm bol}$ as a function of $Ro$ and $B$ in Fig.~\ref{fig:lx}. The X-ray emission relation to Rossby number in our sample follows that of stars that are not known to host close-in planets~\citep{wright2011stellaractivityrotation}, spanning both the saturated and unsaturated regime. The X-ray emission and our estimated magnetic field closely follow each other, also consistent with stars not selected for close-in planets~\citep{reiners2022magnetism}. There are two exceptions: GJ 3082 appears underluminous in X-ray compared to its Rossby number and expected magnetic field strength. AU Mic is slightly overluminous, falling somewhat above the saturated level of $L_X/L_{\rm bol}$. However, the star still follows the $B-L_X/L_{\rm bol}$ relation.  

\begin{figure}[ht!]
    \script{paper_lx_plots.py}
    \begin{centering}
        \includegraphics[width=\linewidth]{figures/PAPER_lxlbol.png}
        \caption{
           X-ray luminosity over bolometric luminosity compared to Rossby number (\textbf{upper panel}), and average surface magnetic field strength (\textbf{lower panel}). Both relations follow those of stars not selected for hosting close-in planets, except for GJ 3082, which is underluminous for its Rossby number, and hence also its estimated magnetic field strength.
        }
        \label{fig:lx}
    \end{centering}
\end{figure}

\subsection{Power of star-planet interaction}
\label{sec:methods:pspi}
We consider two theories for the mechanism behind flaring star-planet interactions, the stretch-and-break and the Alfv\'en wing mechanisms. For both of them, we include the case of a magnetized and an unmagnetized planet, for a total of four estimates of the expected power of magnetic star-planet interaction.

First, we estimate the power of star-planet interaction generated by the stretch-and-break mechanism using the scaling relations in~\cite{lanza2012starplanet} (their Eqn. 45), where $P_{sb}$ is proportional to

\begin{equation}
    P_{sb} \sim B_p^{2/3} R_p^2 B_*^{4/3} R_*^4 v_{rel} a^{-4} .
\end{equation}

$B_p$ and $B_*$ are the planetary and stellar field strengths at their poles, respectively; $v_{rel}$ is the relative velocity between the stellar rotation and the planet's orbit at the semi-major axis $a$; and $R_*$ and $R_p$ are the stellar and planetary radii, respectively. Since the planetary fields are unknown, we ignore them in the calculation. We also do not know the stellar field strength at the pole for the vast majority of the stars, so we adopt the surface field strength~(Section~\ref{sec:methods:bfield}) as a proxy.

Second, we also consider the case of an unmagnetized planet with $B_p=0$, in which case Eq.~45 in~\cite{lanza2012starplanet} reduces to

\begin{equation}
    P_{sb0} \sim R_p^2 R_*^4 B_*^2 v_{rel}    a^{-4} 
\end{equation}

Third, we estimate the power of star-planet interaction generated by the Alfv\'en wing mechanism ~\citep{saur2013magnetic,kavanagh2022radio}, combining Eqns.~8~and~11 in \citet{kavanagh2022radio} to arrive at

\begin{equation}
    P_{aw} \sim B_p^{2/3} R_{p}^2    B_*^{1/3}  v_{rel}^2 a^{-2}  
\end{equation}
where we replace the stellar wind density $\rho_W$ with its proportionality to $a^{-2}$, and stellar wind magnetic field at the planetary radius $B_W$ with the surface field strength $B_* a^{-3}$, which result in the total $a^{-2}$ factor in the above equation.

And fourth, we also consider the case of an unmagnetized planet with $B_p=0$, in which case Eq.~8 in \cite{kavanagh2022radio} reduces to

\begin{equation}
    P_{aw0} \sim R_p^2  B_*  v_{rel}^2  a^{-4}
\end{equation}

Lacking $B_p$, and various additional factors that appear in $P_{sb},P_{sb0},P_{aw}$, and  $P_{aw0}$, such as the angle between the relative velocity and magnetic field of the stellar wind~\citep{kavanagh2022radio}, or the force-free parameter~\citep{lanza2012starplanet}, we restrict ourselves to differential estimates, and normalize all $P_{xx}$ to AU Mic.

\section{Results}
\label{sec:results}

Our goal was to measure magnetic star-planet interaction as the statistical clustering of flares in phase with the innermost planet's orbit. We searched all star-planet systems that were observed with Kepler and TESS at 1 or 2 min cadence, respectively, for flares. For the star-planet systems in the resulting flare catalog~(Section~\ref{sec:results:catalog}), we confirmed that the orbital phase could be known well enough for the entire observing baseline of the system~(Section~\ref{sec:results:coherence}). We calculated the orbital phases of each flare, and the flare phase distributions for each system~(Section \ref{sec:results:phasedist}). We then applied the custom Anderson-Darling test to assess how much each distribution was different from randomly distributed intrinsic flaring. Comparing the test results with the expected power of magnetic star-planet interaction, we found that the amount of excess flaring tentatively increases with the expected power in all considered scenarios~(Section~\ref{sec:results:spi}). However, not all systems with high expected power show signs of flaring SPI, which creates two branches in the distribution of test results. Finally, in Section~\ref{sec:results:individualstars}, we provide additional context for the most interesting systems in our sample, and explain why we had to exclude others from the analysis.

\subsection{Flare catalog}
\label{sec:results:catalog}
We searched a total of 3032 Kepler Quarters and 4181 TESS Sectors of a total of 1811 star-planet systems for flares. We inspected all candidates manually, and added flares observed by K2 on TRAPPIST-1~\citep{paudel2018k2}, and flares from the planet hosting primary in the Kepler-411 binary~\citep{jackman2021stellara}. The final table contains a total of \input{output/PAPER_total_number_of_flares.txt}flares in \input{output/PAPER_total_number_of_systems_with_flares.txt}systems. In Table~\ref{tab:flares}, we list the flares and their characteristics. To identify each event, we provide the name of the system in which the flare was found, along with its TIC, whether it was observed with Kepler or TESS, and during what Quarter or Sector, respectively. For each flare, we also give the start and finish time, relative amplitude $a$ and equivalent duration $ED$. If the transit midtime is known for the innermost planet, we also give the orbital phase at which the flare occurred. The full table is provided with the supplementary material, and via Zenodo (see Data Availability Statement).

\begin{table*}
    \script{paper_latex_flare_table.py}
    \centering
            \caption{
            Flare catalog of all star-planet systems observed by Kepler and TESS (as of July 2022). In transiting multi-planet systems, the orbital phase refers to the innermost planet, with the transit mid-time at phase zero. The full catalog is available online (see Data Availability Statement).
        }
    \input{output/flare_table.tex}
        \label{tab:flares}
\end{table*}

\subsection{Period coherence times}
\label{sec:results:coherence}
In many cases, the observing baseline covered by Kepler and TESS for a given system can span multiple years. In these cases, the orbital period of the innermost planet must be known very precisely so that we can assign accurate orbital phases to the flare events.
We test this by dividing the timespan $\Delta T$ between the first and the last flare in the combined Kepler and TESS observations by the coherence time $\tau$ of the orbital period for each system. We calculate the coherence time as

\begin{equation}
    \tau = P^2 / \sigma_P,
\end{equation}
where $P$ is the orbital period, and $\sigma_P$ is its uncertainty.
If the orbital period is well-known, the resulting ratio $\Delta T/\tau$ should be $<<1$. The bottom panel in Figure~\ref{fig:coherence_hist} shows a histogram of $\Delta T/\tau$ in our sample. The orbital period is known sufficiently well for our analysis in most cases with ratios $<0.02$, that is, the orbital phase of the last flare is at most $2\%$ off, compared to the first. The three cases where the ratio is highest with $0.05$, $0.12$, $0.24$ deviation are TAP 26, GJ 3082, and GJ 674, respectively. We keep these systems in our sample, but caution that the measured absence of flaring SPI might be due to the uncertain orbital period of the planet in these cases.

We repeated the coherence calculation for $P_{rot}$, and found that the ratio is $>0.1$ for most stars~(Figure~\ref{fig:coherence_hist}, top panel), so that a statistical analysis of rotational flare modulation is not feasible.

\citet{fischer2019timevariable} argue that the synodic period, at which the sub-planetary point crosses the same (magnetic) surface element on the star, may result in a clearer interaction signal, because a magnetically active region that is prone to planet-induced flares, will be passed only once per that period. However, the uncertainty in $P_{\rm rot}$ does not allow us to search for modulation with synodic period. We argue, however, that in a transiting system with high orbital inclination, the effect with synodic period reduces to modulation with the orbital period:

If the orbital axis inclination is high, then the passage will regularly occur on the back of the star relative to the observer. The true period will therefore be additionally modulated by the visibility, resulting in an intermittency of the signal. As a consequence, to detect a modulation with the synodic period, a significantly longer observing baseline is needed than to detect a modulation with the orbital period. Additionally, if there are many active regions (at the same time, or in sequence) that produce flares when the sub-planetary footpoint passes across, the effect with the synodic period further converges to a modulation with orbital period. 
In our sample, the observing baseline is multiple years long for many systems. Most stars, even the most rigidly rotating late M dwarfs, show variability in their spot modulation on such long timescales~\citep[e.g.,][]{giles2017kepler,davenport2015spots, namekata2019lifetimes}).

% Even if differential rotation was absent and active regions were extremely stable, we would likely be unable to observe the passage of the sub-planetary point over the flare-prone region every synodic period. 

%However, this assumes that the landscape of active regions on the star is stable throughout the entire observing time, and that the number of such regions in the footpoint passageway is low enough to produce a periodic modulation. Note that this is a different effect from the geometrical modulation of visibility we look for in modulation with the orbital period in this study.

%Even for the few stars, for which the rotational period is known precisely enough, we need to make numerous assumptions. These include the number of active regions crossed by the footpoint of interaction, differential rotation, spot evolution, and a significant correlation between the location of the planet and the location of the planet-induced flare on the stellar surface. For modulation with the orbital period, we require only that last correlation between footpoint location and orbital phase.

%There are three reasons for this: First, the uncertainty for the rotation was not always provided, in which case we assigned a $10\%$ error. Second, rotation period is more difficult to ascertain from a light curves' variability than the orbital period, particularly if the modulation is weak and/or rapidly evolving. And third, differential rotation causes spots at different latitudes to move at different speeds, so it becomes difficult to even define a rotation period, as long as we don't know at which latitude(s) the flares occur~(see also Section~\ref{sec:discussion:rotsyn}).

\begin{figure}[ht!]
    \script{paper_coherence_histogram.py}
    \begin{centering}
        \includegraphics[width=\linewidth]{figures/PAPER_coherence_histogram.png}
        \caption{
           Time span of observation vs. coherence time of the rotational (\textbf{upper panel}) and orbital (\textbf{lower panel}) periods, respectively. Orbital periods are typically known precisely enough, so that the phase uncertainty at the last observed flare is of the order of $10^{-2}$. In contrast, rotation periods are usually less well constrained, so that the phase of the last flare is often undetermined (ratio on x-axis $\geq 1$).
        }
        \label{fig:coherence_hist}
    \end{centering}
\end{figure}

\subsection{Flare phase distributions}
\label{sec:results:phasedist}
Our custom Anderson-Darling test compares the measured distribution of orbital phases of the flares to the expected distribution, and returns the significance of the deviation between the two. In the expected distribution, the same number of flares would be distributed randomly across all phases, as the overwhelming majority of studies looking for variation with rotational phase suggest~\citep[see, e.g.,][]{doyle2018investigating,howard2021evryflare}. Figures~\ref{fig:cumdist_transiting} and \ref{fig:cumdist_rv} show the flare phase distributions along with the corresponding expected distributions for transiting and non-transiting systems, respectively. The expected distributions usually deviate from a straight line because we take into account the coverage of the orbital phases by the Kepler and TESS observations, as well as the different flare rates in each light curve, which arise due to varying noise levels between individual Quarter/Sectors, and Kepler and TESS.

\begin{figure*}[ht!]
    \script{paper_cumdist_individual_transiting.py}
    \begin{centering}
        \includegraphics[width=\linewidth]{figures/PAPER_flares_phase_hist_transiting.png}
        \caption{
            Cumulative distributions of orbital phases of flares in the \textit{transiting} planet hosts observed by Kepler and TESS, sorted by number of flares from top to bottom. The bisector line is dotted, the expected distribution is solid blue, and the observed distribution is solid black. Phase zero corresponds to the transit mid-time of the planet. 
        }
        \label{fig:cumdist_transiting}
    \end{centering}
\end{figure*}

\begin{figure*}[ht!]
    \script{paper_cumdist_individual_rv.py}
    \begin{centering}
        \includegraphics[width=\linewidth]{figures/PAPER_flares_phase_hist_rv.png}
        \caption{
            Cumulative distributions of orbital phases of flares in the \textit{non-transiting} planet hosts observed by Kepler and TESS, sorted by number of flares from top to bottom. The bisector line is dotted, the expected distribution is solid blue, and the observed distribution is solid black. Phase zero is chosen arbitrarily. 
        }
        \label{fig:cumdist_rv}
    \end{centering}
\end{figure*}

\subsection{Flaring star-planet interaction signal}
\label{sec:results:spi}
We used the custom Anderson-Darling test introduced in Section~\ref{sec:methods:adtest} to look for flaring SPI in all systems with three or more flares with equivalent duration $ED>1\,$s, detected in their Kepler and TESS observations. We choose the $ED$ cutoff to make sure that we are comparing similar flares regardless of spectral type, that is, flares above the same energy relative to stellar luminosity, without losing too many flares. In~\citet{ilin2022searching}, AU Mic appeared to be more modulated with orbital period in the high energy flares above $1\,$s than below. It is also the only star in our sample, for which this threshold makes a difference in the significance of the Anderson-Darling test. However, we note that we lose four systems by applying this threshold -- GJ 393, Kepler-411, Kepler-138, and Kepler-1084. We note that the flares (with $ED<1\,$s) in all of these systems are consistent with random flare phases within $1\sigma$.

Table~\ref{tab:maintable_der} lists the $p$-values for each star-planet system. There is no star-planet system for which we find a $>3\sigma$ signal of star-planet interaction in our sample. However, overall, the significance of the deviation increases with higher expected power $P_{xx}$ of interaction, where the subscript denotes different scenarios introduced in Section~\ref{sec:methods:pspi}. Table~\ref{tab:maintable_der} also lists the derived parameters required in these scaling laws, i.e., Rossby number $R$o, surface-average magnetic field strength $B$, and relative velocity between the planet and a co-rotating magnetic field $v_{\rm rel}$. 

Figure~\ref{fig:adtest_bp} shows that in every scenario we considered (Alfv\'en wing or stretch-and-break mechanism, with or without planetary magnetic field), there are two branches on the high end of expected powers. On one branch, the significance of the measured interaction is correlated with the expected power $P_{xx}$. On the other branch, no interaction is measured regardless of the expected power $P_{xx}$. 



\begin{table*}
\footnotesize
\movetableright=-20mm
\script{paper_main_table.py}
\caption{Flaring star-planet interaction. $Ro$, $B$, and $v_{rel}$, are derived from literature values (Table~\ref{tab:maintable_lit}). $P_{xx}$ stands for the power of stretch-and-break ($sb$) and Alfv\'en wing ($aw$) interaction mechanisms, assuming the planet has a magnetic field strength of 1 G. $P_{xx0}$ is the same, but assuming an unmagnetized planet. All powers are normalized to AU Mic. The $p$-value of the Anderson-Darling test is lower when the system shows more flares periodic with the planetary orbit.}
\input{output/table_der_vals.tex}
    \label{tab:maintable_der}

\end{table*}


\begin{figure*}[ht!]
    \script{paper_adtest_vs_value_scatterplots.py}
    \begin{centering}
        \includegraphics[width=\linewidth]{figures/PAPER_ADtest_bg.png}
        \caption{
            Expected power of SPI vs. AD test results, assuming four different scenarios, color-coded by stellar surface field strength. Top row: Stretch-and-break scenario. Bottom row: Alfv\'en wing scenario. Left column: assuming $B_p=1\,$G. Right column: assuming $B_p=1\,$G. While the distribution of $p$-values is consistent with no interaction, all scenarios indicate lower $p$-values only for high expected powers of interaction. See Fig.~\ref{fig:adtest_ro} in the Appendix for the same figure, color-coded by Rossby number.
        }
        \label{fig:adtest_bp}
    \end{centering}
\end{figure*}



\subsection{Individual stars}
\label{sec:results:individualstars}
The properties of star-planet systems in this work are diverse, including both very fast and very slowly rotating host stars; systems with super-Earths, Neptunes, and Hot Jupiters; and spectral types covering the lower main sequence from mid-F to late M. 

For our analysis, it is important to know that the observed flares indeed occurred on the planet host star. We therefore excluded some systems due to documented contamination from nearby (bound or co-moving) companions~(Section~\ref{sec:results:individualstars:excluded}). One exception to this rule is Kepler-411, a binary system for which the flare contributions from each component could be separated, but which was excluded for other reasons~(Section~\ref{sec:results:individualstars:kep411}). We also dropped GJ 1061 from the sample because its rotation period was unconstrained. Another uncertainty in our analysis is the possibility of additional planets at even shorter orbits than the currently known innermost planet, such as might be the case for Proxima Cen~(Section~\ref{sec:results:individualstars:proxima}). 

The bulk of the systems in our sample are neither expected to show high power of magnetic star-planet interaction, nor do they show any deviation from intrinsic stellar flaring. In the subsample of stars with high expected power, of the order of AU Mic's and above, we first take a closer look at those that seem to follow an increase in measured flaring star-planet interaction with increasing expected power $P_{xx}$, that is AU Mic itself, K2-25, TOI-540, and HIP 67522~(Sections \ref{sec:results:individualstars:aumic} and \ref{sec:results:individualstars:hip67522}). Then we consider those systems where high power was expected in all scenarios, but not measured, i.e., TAP~26 and KOI-12~(Sections \ref{sec:results:individualstars:tap26} and \ref{sec:results:individualstars:koi12}).
\subsubsection{Excluded targets}
\label{sec:results:individualstars:excluded}

While they appear in the flare catalog~(Table~\ref{tab:flares}), we excluded a number of systems from further analysis. We drop all multiple stars except for Proxima Cen (whose components are well separated both physically, and on the sky), and all stars with nearby objects that contaminate our analysis, e.g., from Gaia DR2 and ground-based adaptive optics~\citep{ziegler2018measuring}: \input{output/multiples_string.tex}

For instance, Kepler-808 (KOI-1300) has a $\sim0.4M_\odot$ \citep{kraus2016impact} companion that is about 1.8 mag fainter than the primary at a separation of .78 arcsec~\citep{baranec2016roboao}. TOI-837 has co-moving M dwarf a few arcsec away in the same young cluster, 5 mag fainter, but the primary is an F9-G0 star, so flares could still originate come from both stars~\citep{bouma2020cluster}. HD 41004 B, DS Tuc A, and LTT 1445 are known multiple systems where all components contribute significantly to the total flux.

The only single star system excluded is GJ 1061. The star is lacking a stellar rotation period estimate, which we need to calculate the magnetic field, and $v_{rel}$. \citet{dreizler2020reddots} estimate that it is a fairly slow rotator with a period between 50 and 200 days, so we do not expect it to have a strong magnetic field anyway.

\subsubsection{Kepler-411}
\label{sec:results:individualstars:kep411}
Kepler-411 has a 3 mag fainter companion~\citep{wang2014influence,ziegler2018measuring}, but we do not exclude it from the analysis initially. The system was observed in Kepler short cadence by \citet{jackman2021stellara} and both the planet host, and the companion star flare. \citet{jackman2021stellara} disentangle the contributions from each component on a pixel level. According to \citet{morton2016false, sun2019kepler411}, Kepler-411's planets orbit the primary companion. The fainter companion of Kepler-411 appears to cause the majority of flares (41), whereas the planet host causes only 7. Adopting the 7 flares from \cite{jackman2021stellara} in our analysis, we find that they all have equivalent duration $ED<1\,$s, which excludes Kepler-411 from further analysis. Note that we do not find the low energy flares to cluster in orbital phase, either.

\subsubsection{Proxima Cen}
\label{sec:results:individualstars:proxima}
We do not detect a deviation from intrinsic flaring on Proxima Cen, consistent with numerical models that place Proxima Cen b well outside the sub-Alfv\'enic zone~\citep{kavanagh2021planetinduced}. However, this might not entirely exclude Proxima Cen from the search for flaring star-planet interaction. The tentatively detected Proxima Cen d, a planet further in, at 0.029 au, or 5 day orbital period~\citep{faria2022candidate, artigau2022linebyline} could still cause flaring star-planet interactions. However, the orbital period of the tentative planet is so uncertain at this point that its coherence time~(Section~\ref{sec:results:coherence}) is shorter than the roughly 2yr observing baseline of Proxima Cen, preventing us from accurately measuring the flare phases. 


\subsubsection{AU Mic, K2-25, and TOI-540}
\label{sec:results:individualstars:aumic}

AU Mic is a 16-29 Myr old pre-main sequence M0-M1 dwarf with a strong magnetic field of about $3010\pm220\,$G obtained from Zeeman broadening measurements~\citep{reiners2022magnetism}. The innermost Neptune-sized planet, AU Mic b, could be both inside the sub-Alfv\'enic zone if the star's mass loss rate is relatively low at about 30 times the solar value or lower. If the mass loss rate is high, exceeding the solar value by a factor of several hundreds, AU Mic b could orbit in the super-Alfv\'enic zone, and become unable to experience any planet-induced flaring. Recently, \citet{klein2022one} found tentative periodicity with the orbital period of AU Mic b in the chromospheric He I D line, which increases if the contribution from flares is included in the calculation. Overall, our marginal signal of flaring star-planet interaction in the AU Mic system supports the idea that this young, magnetically active M dwarf system with a close-in Neptune could exhibit planet-induced flaring.
For a detailed discussion of AU Mic and its flaring star-planet interaction, we refer to~\cite{ilin2022searching}, where we also estimate that an additional 50–100 days of TESS-like monitoring of AU Mic would yield a $3\sigma$ detection if the marginal signal in our data is real.
 
K2-25 is a multiplanet system around a fast rotating ($P_{\rm rot}<2\,$d) mid-M dwarf in the Hyades open cluster~\citep[600-800 Myr, ][]{stefansson2020habitable}. Both the expected power of interaction, and the measured deviation, are remarkably similar to the AU Mic system. Interestingly, K2-25 b is in a presumably similar environment to GJ 436 b, but does not show the atmospheric escape in Ly$\alpha$ that the latter famously exhibits~\citep{rockcliffe2021lya}. This might in part be due to the stellar wind conditions around K2-25 b. Further investigation of magnetic star-planet interactions, flaring or otherwise, in this system could add an important constraint on K2-25's wind properties by clarifying whether the planet orbits in- or outside the Alfv\'en radius, especially accounting for its moderately high eccentricity~(see also Section~\ref{sec:discussion:eccentricity}).

In Figure~\ref{fig:adtest_bp}, TOI-540 clusters with K2-25 and AU Mic -- a fast rotating M dwarf with its innermost planet in a $1.24\,$d orbit~\citep{ment2021toi}. Compared to the other two systems, its innermost planet is not a Neptune, but a rocky planet slightly smaller than Earth~\citep{ment2021toi}. It is a relatively smaller obstacle in the host's magnetic field, yet the interaction is expected to be of a similar magnitude due to the strongest inferred magnetic field, and closest orbit among the three systems. 

In the scenario wherein the stretch-and-break mechanism drives the interaction, the three systems cluster together most distinctly~(top row in Figure~\ref{fig:adtest_bp}). In the Alfv\'en wing scenario~(bottom row in Figure~\ref{fig:adtest_bp}), the distinction is less clear. In both cases, the low level (or possible absence) of interaction could be due to the Alfv\'en radius being within the planetary orbit. It is important to note that the scaling laws introduced in Section~\ref{sec:methods:pspi} assume that the planet is sub-Alfv\'enic, but make no statement about it, because the mass loss of these stars is largely unconstrained. All else equal, a planet is more likely sub-Alfv\'enic if the stellar magnetic field is strong. Another explanation is the possible intermittency, or 'on-off nature' of the interaction~\citep{shkolnik2008nature}, which may be at play here, but particularly in cases where the expected power is very strong, such as in the comparison between HIP 67522 and TAP 26 that we address in the following two Sections.

\subsubsection{HIP 67522}
\label{sec:results:individualstars:hip67522}
HIP 67522 has one of the strongest expected SPI signals in our sample, and shows the clearest sign of flaring SPI at $>2\sigma$ level, albeit with only 6 flares in the sample. These flares are distributed across two Sectors in TESS. In Sector 11, two flares occur at orbital phases $\sim 0.61$ and $\sim 0.81$. Two years later, in Sector 38, four flares occur, but this time, all of them take place at orbital phases $0.00-0.05$, that is, within 10 hours after mid-transit.

HIP 67522 is a young Sun, currently contracting onto the main sequence. It is a Sco-Cen member (10-20 Myr old), which was discovered to host a close-in Jupiter, HIP 67522 b, in 2020~\citep{rizzuto2020tess}. Curiously, the planet is in close spin-orbit commensurability -- $P_{\rm rot}/P_{\rm orb}\approx 1/5$. Therefore, we cannot completely rule out that the observed periodicity is in fact a rotational periodicity, and not related to the planet. However, if we calculate the $p$-value of the AD test using the rotational period instead of the orbital period, we find HIP 67522 consistent with uniform flaring in time within $1\sigma$, which is likely due to the not sufficiently low uncertainty in rotation period. So we cannot rule out rotational modulation yet, although the clustering close to transit is suggestive.% Fortunately, HIP 67522 will be observed again with TESS in April 2023, which may resolve the puzzle.

%It was found with low obliquity, consistent with ~\citep{heitzmann2021obliquity}. Tentative third planet transiting.
%The system's $a/R_*\approx11.7$, eccentricity below .25~\cite{rizzuto2020tess}, so that  
 %\cite{wood2021characterizing} rule out stellar companions for HIP 67522 based on RV, high-resolution imaging, and Gaia data.

\subsubsection{TAP 26}
\label{sec:results:individualstars:tap26}
TAP 26 is a weak-line T-Tauri star with a strong magnetic field~\citep{yu2017hot}. \citet{lanza2018closeby} estimate that the system can release more energy in magnetic star-planet interaction than other  systems with Hot Jupiters, such as HD 179949, that has previously been detected with chromospheric variability in phase with the planet's orbit~\citep{shkolnik2008nature}. However, for TAP 26 b, the orbital period might not be well-constrained. \citet{yu2017hot} apply several methods to derive $P_{\rm orb}$ from the radial velocity data: While the $10.8\,$d orbit we adopted is the most likely according to \citet{yu2017hot}, a $9.0\,$d orbit is also likely, and some of their applied methods favored a $13.4\,$d period. The absence of interaction could hence be a consequence of uncertain orbital period. However, intermittent interactions and the viewing geometry of TAP 26 could also explain the absence of phase-correlated flaring~(see Sections~\ref{sec:discussion:intermittency} and \ref{sec:discussion:viewing}).

\subsubsection{KOI-12}
\label{sec:results:individualstars:koi12}
KOI-12, also known as Kepler-448, is a two-planet system with an eccentric outer planet KOI-12 c, and an inner Warm Jupiter planet KOI-12 b in a $17.8\,$d orbit~\citep{masuda2017eccentric}. With respect to rotation rate, stellar and planetary radius, the KOI-12 system is similar to HIP~67522 and TAP~26 (although for the those young stars the radii are large because the stars are still contracting onto the main sequence). In constrast, KOI-12 is $1.4\pm0.3\,$Gyr old~\citep{bourrier2015sophie}, an F5 subgiant~\citep{frasca2016activity}, which is already evolving off the main sequence. Yet, despite its high expected power of interaction, we detect no excess flaring in phase with KOI-12 b. Here, the reason could be that KOI-12 b is in fact super-Alfv\'enic due to the relatively wide orbit compared to TAP 26 and HIP 67522. Intermittent interaction might play a role here as well, and could be favored if future observations capture episodes during which the interaction is 'on'.



%\subsection{Combined Sample}

%For those stars with given transit mid-time and 

\section{Discussion}
\label{sec:discussion}

Our results in Fig.~\ref{fig:adtest_bp} are tentative, yet the trend between expected power of interaction and measured clustering of flares in orbital phase is suggestive, as is the appearance of an active and an inactive branch at high expected powers. We can now consider additional facts about the systems, as well as processes, that may explain the observed trends. We consider the extent of the Alfv\'en radius that decides over the possibility of magnetic interaction in the first place~(Section~\ref{sec:discussion:as}), and how orbital eccentricity can further enhance magnetic SPI signal~(Section~\ref{sec:discussion:eccentricity}). The observed branching might also be explained by intermittent SPI~(Section~\ref{sec:discussion:intermittency}), but its possible causes are not well known. As an easier to test alternative explanation, we highlight the important influence of viewing geometry on the observability of flaring SPI, which might explain the absence of magnetic SPI in TAP~26~(Section~\ref{sec:discussion:viewing}). 

Magnetic and tidal SPI are difficult to disentangle by their global effects on stellar activity indicators~(Section~\ref{sec:intro:global}). Locally, however, we can discriminate between the two by their relevant periods, that is, $P_{\rm orb}$ and $P_{\rm orb /2}$. In Section~\ref{sec:discussion:tidal}, we repeat the analysis in this paper with $P_{orb}/2$, and find yet more tentative trends in line with scaling laws for tidal interaction. In a system-by-system comparison, we find consistency with the expectation that slowly rotating, likely old, systems can only show local tidal interaction, while rapidly rotating systems with strong magnetic fields can experience both.
\subsection{Alfv\'en surface}
\label{sec:discussion:as}

\begin{figure}[t]
    \script{paper_as_vs_ad.py}
    \begin{centering}
        \includegraphics[width=\hsize]{figures/PAPER_AS_vs_AD.png}
    \caption{ TBD.}
        \label{fig:as}
    \end{centering}
\end{figure}

reference \ref{sec:intro:mspi}

Fig.~\ref{fig:as}

how we estimated the AS (Judy's part)


what we think about it (E and J contributions, comments from others  welcome)


\subsection{Eccentricity}
\label{sec:discussion:eccentricity}
Orbital eccentricity can affect magnetic SPI by modulating the planet's distance to the star. If the planetary orbit is highly eccentric, the difference between periastron and apoastron can be significantly larger than the variability of the Alfv\'en radius. If then apoastron is outside this radius, and periastron within, the planet may find itself in the sub-Alfv\'enic zone only for a short amount of time, during the planet's rapid periastron passage. This was previously exploited in the case of HD 17156~\citep{maggio2015coordinated}, and analogously for the colliding magnetospheres in binary systems~\citep{massi2002periodic,getman2016search}. In our sample, the estimates for eccentricities of the innermost planets are all moderate to low, with $e\leq 0.35$. The exception is K2-25, with a relatively high $e=0.45$. If periastron passage in K2-25 occurs when the planet is in front of the star relative to the observer, the visibility effect of the interaction footpoint is further increased by the narrow phase range of the periastron passage. \citet{stefansson2020habitable} estimated the argument of periastron $\omega=120^{+12}_{-14}\,$deg for K2-25 (with longitude of the ascending node chosen such that $\omega$ is the same as the longitude of periastron, with $\omega=90\,$deg being periastron passage during transit,~\citealt{kipping2010investigations,dawson2012photoeccentric}). We can therefore conjecture that the tentative signal of flaring SPI in K2-25 is enhanced by eccentricity. Statistically, we are more likely to observe a planetary transit close to periastron, since periastron permits more orbital inclinations than apoastron. So, overall, we expect eccentric transiting systems to experience elevated SPI signal.

% \subsection{Rotational and synodic variability}
% \label{sec:discussion:rotsyn}
% Our analysis is the most comprehensive search for flaring star-planet interactions to date, using the full Kepler and most recent TESS archives of light curves. We considered exploiting it beyond the search for variations with orbital period, and additionally look for variability with rotation period, and synodic period. Here, we explain why we ultimately decided not to do so.

% Even for the few stars, for which the rotational period is known precisely enough~(see Fig.~\ref{fig:coherence_hist}) for our analysis, we need to make numerous assumptions. These include the number of active regions crossed by the footpoint of interaction, differential rotation, spot evolution, and a significant correlation between the location of the planet and the location of the planet-induced flare on the stellar surface. For modulation with the orbital period, we require only that last correlation between footpoint location and orbital phase. Any realistic number of active regions (one or more), a wide range of degrees of differential rotation (from rigid rotator to solar-type), and half-life duration of active regions (day to years) are acceptable, because the observed effect depends predominantly on the visibility of the footpoint, which is governed by orbital motion. The latter is much more stable over long periods of time than the intrinsically dynamic magnetic field configuration~(see e.g., various works on spot lifetimes in low mass stars~\citealt{giles2017kepler,davenport2015spots, namekata2019lifetimes}).

% \citet{fischer2019timevariable} argue that the synodic period, at which the sub-planetary point crosses the same (magnetic) surface element on the star, may result in a clearer interaction signal, because a magnetically active region that is prone to planet-induced flares, will be passed only once per period. However, this assumes that the landscape of active regions on the star is stable throughout the entire observing time, and that the number of such regions in the footpoint passageway is low enough to produce a periodic modulation. Note that this is a different effect from the geometrical modulation of visibility we look for in modulation with the orbital period in this study.

% In our sample, the observing baseline is multiple years long for many systems. Even the most rigidly rotating late M dwarfs show variability in their spot modulation on such long timescales. Even if differential rotation was absent and active regions were extremely stable, we will not be able to observe the passage of the sub-planetary point over the flare-prone region every synodic period. If the planet's rotational axis inclination is high, then the passage will regularly occur on the back of the star relative to the observer. The true period will therefore be additionally modulated by the visibility, resulting in an intermittency of the signal. As a consequence, to detect a modulation with the synodic period, a significantly longer observing baseline is needed than to detect a modulation with the orbital period. Additionally, if there are many active regions that produce flares when the sub-planetary footpoint passes across, the effect with the synodic period is further diluted. 


\subsection{Intermittency}
\label{sec:discussion:intermittency}
Magnetic star-planet interaction power in a given system may sometimes be low for episodes of time, and high for others. HD 179949 is the prototype example for this phenomenon. The orbital modulation of its chromospheric indicators was observable for four out of six epochs on a five-year baseline~\citep{shkolnik2003evidence,shkolnik2008nature}. It is not clear how long a continuous epoch of interaction or non-interaction might be. This leaves room for many different explanations. Uninterrupted, long-term monitoring of individual systems is required to investigate the reasons for the temporary cessation of interaction. Candidate mechanisms for the star include: 
\begin{itemize}
    \item the Alfv\'en radius moving in- and out the planet's orbit, either over the course of an activity cycle, or through short-term variations in the magnetic field and wind properties caused by, e.g., coronal mass ejections;
    \item the interacting footpoint moving to higher latitudes due to changes in the large scale field, which reduces the modulation of its visibility 
    \item an active latitude (i.e., small scale field) moving away from the magnetic footpoints passageway; or
    \item the available magnetic energy decreasing such that the flares or other magnetic interaction indicators still occur but fall below the detection threshold for a period of time.
\end{itemize}

If the magnetic field of the planet also changes over time, this might modulate the intensity as well~\citep[e.g.][]{turnpenney2018exoplanetinduced}.

The location of active regions within the passageway of the magnetic footpoints may be such that the interaction occurs only when the region faces away from the observer. This could span multiple orbits of the planet, particularly when the surface is populated by few stable active regions, and the orbital and rotational periods are similar. The flip-flop behavior of active longitudes seen in young solar-like stars~\citep{berdyugina2005starspots} could then cause a relatively sudden quenching of interaction by moving the interacting region out of view for a period of time. 

In our results in Fig.~\ref{fig:adtest_bp}, we can interpret the two branches emerging at high expected power of interaction as due to intermittency. TAP 26 and KOI-12 do not show any sign of excess flaring in our data, despite similar expected powers as in HIP 67522 with its $2-3\sigma$ signal. However, other explanations cannot be ruled out yet (see Sections~\ref{sec:results:individualstars:tap26} and \ref{sec:results:individualstars:koi12}, and the following).

%In Sections~\ref{sec:results:individualstars:tap26} and \ref{sec:results:individualstars:koi12}, the absence of interaction in TAP 26 and KOI-12 could be explained by an uncertainty in orbital period that smears out the signal, and a super-Alfv\'enic orbit, respectively. Yet intermittency cannot be ruled out. The case of HIP~67522 may be more informative. In our data, we combine the observation from two TESS Sectors that are almost 2 years apart. The four phase-correlated flares all occur in the later Sector, while the earlier one contains only two flares, both of them at orbital phases different both from each other and the four phase-aligned flares~(see Fig.~\ref{fig:cumdist_transiting}). This can be interpreted as a tentative sign of intermittency of flaring star-planet interactions in this system. 

\subsection{Viewing geometry}
\label{sec:discussion:viewing}
Modulated visibility of the magnetic interaction footpoint is crucial for the detection of flaring star-planet interactions in our analysis. The viewing geometry of the star-planet system is therefore important to consider as an alternative to, or explanation for, the intermittency discussed above. Assuming for the sake of the argument that the sub-planetary point is close to the magnetic interaction footpoint in longitude, two factors will determine whether the footpoint's visibility will be modulated with the planetary orbit or not: the orbital inclination and the footpoint's latitude relative to it. 

Consider a system, where orbit, and stellar spin and stellar magnetic dipole are aligned. If the footpoint's latitude is low, i.e., close to the orbital plane, the orbital plane can be inclined to a high degree, and the footpoint will still periodically move in and out of view. However, if the footpoint's latitude is high, we need to observe the star-planet system nearly edge-on to be able to measure a modulation of activity with the planet's orbital phase. Assuming the dipole field dominates at the distance of the planet, the footpoint will be close to the pole, similar to the UV spot created by Io and the other moons in the polar cap of Jupiter~\citep{clarke1996farultraviolet, prange1996rapid}. In this scenario, many, especially non-transiting, systems may experience flaring star-planet interaction, but those would not show up as correlation of flare timing with the orbital phase in our data.

Most of our systems are transiting, and for those we can expect this scenario to be a minor problem unless the latitude of the footpoint is very high. Of the systems that have high expected power but do not show any excess flaring, only TAP 26 b is a planet detected in radial velocity, and has a relatively high orbital inclination of about $55\pm10$ deg inferred from photometry, spectroscopy, and Zeeman Doppler measurements~\citep{yu2017hot}. Their ZD map reveals magnetic field concentrations close to the rotational pole, and the brightness maps also show a dark spot in the same location. If the orbital plane is aligned with the stellar rotation axis, the absence of phase-correlated flaring in TAP 26 could be explained by the footpoint of interaction being always in view, and therefore never modulated. 

However, alignment between orbital, rotational and magnetic dipole axis is not generally given. Many planets orbit their hosts nearly pole-on~\citep{albrecht2012obliquities, albrecht2022stellar, bourrier2023dream}. T Tauri stars show somewhat misaligned magnetic axes relative to their spin~\citep{mcginnis2020magnetic}. In Sun-like stars, the magnetic dipole axis location changes throughout the activity cycle~\citep{petit2009polarity,borosaikia2018direct}, which may also cause temporary cessation of visibility. Orbital precession of misaligned planets can also cause such intermittency, if it periodically brings the footpoints trajectory on the stellar surface fully into view so that it is no longer modulated. 

%Interpreting non-detections as effects of viewing geometry in fully aligned systems ignores this caveat.  

%If we assume that the magnetic dipole axis and the rotation axis of the star are relatively well-aligned, the spin-orbit alignment may have an influence on interaction strength. For instance,  HAT-P-11 rotates nearly pole-on, and HAT-P-11 b is transiting in a near-polar orbit~\citep{bourrier2023dream}.

% \subsection{(Mis-)alignment between orbital, rotational and stellar magnetic axis}
% \label{sec:discussion:misalignment}

%  Relaxing the assumption of alignment between rotational and magnetic dipole axis, we introduce a further complication that is not only due to viewing geometry. 

% or more broadly system architecture including misalignment of the magnetic axis. See HAT-P-11, or instance, and the many polar orbits in \cite{bourrier2023dream}

% \cite{miskovetz2022resolving}  found background stars that were not co-moving HAT-P-11.

% HD 179949 has a strong misalignment of about 70 deg between rotational and magnetic axis. Time-resolved X-ray observations~\citep{acharya2022xray} suggest that the magnetic interaction footpoint is close to the stellar rotational pole, which would be close to the magnetic equator of the strongly dipolar field of HD 179949




\subsection{Tidal interaction}
\label{sec:discussion:tidal}
\begin{figure*}[t]
    \script{paper_tidal_interaction.py}
    \begin{centering}
        \includegraphics[width=\hsize]{figures/PAPER_tidal_AD_test.png}
    \caption{Deviation from random flare times in phase with $P_{\rm orb}/2$, i.e., in phase with tidal bulges raised on the star, compared to different models of tidal interaction~(Section~\ref{sec:discussion:tidal}).
           \textbf{Left panel}: The tidal torque in the system for stars with a convective envelope. Blue color indicates that the star rotates slower than the planet orbits, so that angular momentum is transferred from the orbit to the spin of the star. Grey color is the reverse. \textbf{Middle panel}: The tidal dissipation timescale. \textbf{Right panel}:  Relative gravitational perturbation in the star.}
        \label{fig:tidal}
    \end{centering}
\end{figure*}

\begin{table*}
\footnotesize
\movetableright=-20mm
\script{paper_table_tidal.py}
\caption{Flaring tidal star-planet interaction parameters and $p$-value of the custom Anderson-Darling test. }
\input{output/table_tidal.tex}
    \label{tab:tidal}
 \tablerefs{\input{output/table_tidal_bibstring.tex}}
\end{table*}

\subsubsection{Locally expressed tidal interaction}
Tidal interaction can also lead to enhanced activity, globally, and locally.  Locally, we expect to see a periodicity with half the orbital period of the planet, corresponding to the two tidal bulges forming in the stellar envelope~\citep{cuntz2000stellar}. \citet{holzwarth2003dynamics} suggest an asymmetry induced in the stellar dynamo by the tidal disturbance, which leads to a preference for active (flaring) regions at longitudes about $P_{\rm orb}/2$ apart from each other. 

No observations of local tidal star-planet interaction exist yet. However, in the extreme case of RS CVn binaries, starspots often appear modulated with half the orbital period of the system~\citep{olah2002time, ozavci2018recurrent, kriskovics2023ei}. A similar effect was measured when the companion was likely a very low mass star or a brown dwarf~\citep{donati1995activity, frasca2008spots, parks2021interferometric}, and in the case of a white dwarf companion~\citep{hussain2006spot, watson2007roche}. In ellipsoidal binaries, a modulation of flares with $P_{\rm orb}/2$ beyond geometrical effects was observed with Kepler~\citep{gao2016whitelight}, suggesting flaring regions at the location of the extreme tidal bulges in these stars. However, all these systems are synchronized, $P_{rot}=P_{\rm orb}$, so that intrinsic active longitudes~\citep{usoskin2007longterm, weber2013theory, jarvinen2005spots, lanza2009corot} cannot be disambiguated from the result of tidal star-star interaction ~\citep{holzwarth2003dynamics}. 

If the observed effects are due to the tidally interacting stellar companion, planetary companions should do the same, albeit at a smaller magnitude. Fortunately, spin and orbit are not synchronized in our sample, so confusion with intrinsic effects is less likely. 

\subsubsection{Tidal interaction scaling laws}

We cannot tell whether an additional individual flare is produced by tidal or magnetic interaction, but we can look for deviations from a random distribution of flare times in phase with the relevant period, i.e., $P_{\rm orb}/2$. 

We applied the same technique as described in Section~\ref{sec:methods:adtest} to determine the significance of a local tidal SPI signal. Since we lack a model to relate tidally induced flaring to the systems' properties, we follow~\citet{ilic2022tidal}, and compare the resulting $p$-values with three different scaling laws for tidal interaction as proxies for expected excess flaring due to tides in Fig.~\ref{fig:tidal}. We list the derived values in Table~\ref{tab:tidal}. 

The first model measures the torque $\partial L_{\rm conv} / \partial t$, exerted by angular momentum transfer between star and planet, assuming the star has a convective envelope~(\citealt{penev2012constraining}, their Eqns. 1 and 2). We choose the tidal quality factor $Q_*=10^7$. Our results in Fig~\ref{fig:tidal} (left panel) do not depend on the choice of $Q_*$, as we only need to know the tidal torque down to a proportionality factor. We also make the simplifying assumption that the rotation period of the convective envelope is the same as the surface rotation period measured as described in Section~\ref{sec:data:rotationperiods}, ignoring differential rotation effects.

The second model considers the tidal dissipation timescale -- the shorter the timescale, the higher the power of the interaction~\citep{zahn1977tidal}. Following~\citet{ilic2022tidal}, we adopt Eqn. 2 in~\citet{albrecht2012obliquities} for the tidal dissipation timescale in the convective envelope. We assume that all stars in our sample have a convective envelope because they flare, and flaring requires a dynamo-generated magnetic field, which can only operate in stars with convective envelopes. Most of our stars are dwarf stars below the Kraft break at spectral type F5~\citep{kraft1967studies}, except for KOI-12, which is an F5 sub-giant~\citep{frasca2016activity}.

The third model uses the gravitational perturbation as a proxy for the interaction~\citep{cuntz2000stellar}, originally derived for star-planet interactions.  Their Eqn. 1 proposes a gravitational perturbation proportional to the mass ratio between planet and star, and $a^{-3}$. 

\subsubsection{Planetary and stellar masses}

We collect the planetary and stellar masses from the literature~(Table~\ref{tab:tidal}). For missing planetary masses, the radii were known, so we used the mass-radius relations implemented in \texttt{astro-forecaster}~[Ben Cassesse's implementation of \texttt{forecaster}][]\citep{chen2017probabilistic}. GJ 3323, GJ 674 and GJ 3082 were also missing stellar mass estimates, which we obtained from the \citep{mann2015how, mann2016erratum} relations between absolute Ks magnitude~\citep{skrutskie2006two} and stellar mass for M dwarfs using the distances from \citet{bailer-jones2021estimating}. For Kepler-42 b, we used  $M_p=0.1-2.06M_\oplus$ that covers pure rock to pure iron compositions in \citet{muirhead2012characterizing}, and adopted the logarithmic mean for our estimate. Kepler-1558 c has a similar radius to Kepler-42 b, i.e., most likely a bare core planet, so we used the same density range to estimate its mass. We also adopt the upper limit of $5M_J$ for HIP 67522, but use the lower error from \citet{chen2017probabilistic}. 

\subsubsection{Tidal vs. magnetic flaring star planet interaction}

We find that no single system shows significant signs of tidal star-planet interaction. Overall, the signal in phase with $P_{\rm orb}/2$ is smaller than with $P_{\rm orb}$. However, in all three scenarios, the sample's deviation from random flare timing increases with expected tidal interaction strength. We can compare the magnetic interaction measurements to the tidal ones in some of the conspicuous systems:

\paragraph{HIP 67522} appears with tentative signs of both tidal and magnetic interaction (measured and expected in the different scenarios). On the one hand, since the flares in HIP 67522 mostly appear close to the transit phase, and not during eclipse, this may favor magnetic interaction, and the tidal signal would only be an artifact. On the other hand, the narrow range in which most of the flares on HIP 67522 occurred~(see Fig.~\ref{fig:cumdist_transiting} and Section~\ref{sec:results:individualstars:hip67522}) might favor a well-localized active region in the vicinity of the tidal bulge. Conceivably, both effects play a role at the same time.

\paragraph{KOI-12,} which lacks signal with $P_{\rm orb}$ despite high expected power of magnetic interaction, shows elevated expected and measured tidal interaction. This could be interpreted as KOI-12 b orbiting outside the Alfv\'en radius, so that it can still interact tidally, but not magnetically. However, we note that KOI-12 may be evolving off the main sequence, so that the tidal interaction scaling laws may not apply for this star. 

\paragraph{TAP 26,} which shows no signs of magnetic SPI, does not show any sign of tidal interaction, either, despite both being high according to the scaling laws. This could again be due to the inclination of TAP 26's orbit, but only if the flares caused by the tidal bulge are not close to the equator. Otherwise, TAP 26's tidally induced flares should be more modulated than the magnetically induced ones.

\paragraph{Kepler-42 and TOI-540} Kepler-42 is neither expected to interact magnetically, nor did it show any deviation from random flare timing in phase with $P_{\rm orb}$. However, since it is the system with the shortest planetary orbit in our sample, it is both expected to and does show marginal signal of tidal interaction. Kepler-42 is very similar to TOI-540, which also is a small M dwarf with a likely terrestrial planet in a very short orbit. A key difference is that Kepler-42 is rotating at about 70 days, and therefore expected to have a relatively weak magnetic field, whereas TOI-540 is a young star with $P_{\rm rot}<1\,$d. TOI-540 is expected to have lower tidal interaction strength than Kepler-42 due to TOI-540 b's wider orbit, and also shows a weaker tidal flaring SPI signal in all three scenarios. 

\paragraph{K2-25 and AU Mic} cluster with TOI-540 in magnetic SPI -- all three show tentative signs of magnetic interaction and similar expected powers. But in contrast to TOI-540, both K2-25 and AU Mic are consistent within $1\sigma$ with no observable tidal interaction.

Overall, tidal interaction, if at all present, is less pronounced than magnetic SPI in our sample. However, if the observed deviations from intrinsic flaring do in fact represent low level tidal interaction signal, we can conclude that close-in planets can interact both magnetically and tidally. In line with theoretical considerations~\citep{strugarek2017fate}, our results tentatively suggest that magnetic interaction is more pronounced for planets with fast rotating, active hosts, while tidal interactions affect both systems with slowly and rapidly rotating stars.

%The results here, as with magnetic SPI, are suggestive, as they are consistent with theoretical expectation, albeit at low significance. Further observations, underway with TESS and soon PLATO, may confirm or reject the trends seen for both types of interaction. As multiple observations can be combined regardless of the gaps in time between individual campaigns with our method, we will become able to sensitively constrain the magnitude of interaction, and the relative contributions from tidal and magnetic SPI.

%\subsection{Spin-orbit commensurability}
%\label{sec:discussion:commensurability}
%Curiously, in our analysis, systems with high expected power are divided: Stars that show a correlation between excess flaring and expected power tend to have closer spin-orbit commensurability than systems that are far away from any low-integer ratio between rotational and orbital period.

%In Fig.~\ref{fig:spinorbit}, we split our sample into three categories. The first contains the systems where the expected interaction is lower than that in TOI-540, the system with the lowest power of interaction among those with relatively high flaring star-planet interaction signal (i.e., K2-25, TOI-540, HIP 67522, and AU Mic). The second includes the systems where a high power of interaction is expected, but none is measured (e.g., TAP 26, KOI-12). The final third category contains all system where the data suggest a correlation between expected power of interaction, but do not belong in the first category. We search for spin-orbit commensurability with integers $<10$ in each system, and plot the relative difference between the closest ratio and the actual spin-orbit ratio. Systems with low expected power of interaction span a wide range of spin-orbit commensurability. However, the second and third group form two separate populations. Non-interacting systems are further away from low-integer commensurability, than interacting ones, even though both have high expected power. A similar picture arises when the same sample split is applied to the other interaction scenarios. 

%While chance alignment is possible, the division suggests that spin-orbit commensurability and magnetic star-planet interaction are related. Either something about our observing method favors the measurement of excess flaring in systems with close spin-orbit commensurability, or there is physical causation at play. 

%In the latter case, the spin-orbit commensurability could be either an effect, or a cause, of measured magnetic star-planet interaction. \citet{lanza2022model} suggest that tidal torque of the planet on the star excites resonant oscillations in the interior stellar magnetic field, which in turn leads to a resonance between stellar spin and planetary orbit. These oscillations can lead to a surface pattern of emerging magnetic field that rotates with the spin-orbit commensurable period. Systems with spin-orbit resonance would then show clearer interaction signal, because the resonance would stabilize the pattern of regions with strong magnetic field that favor planet-induced flaring. \citet{szabo2021changing} conjecture that many systems within the sparsely populated region of close-in planets around rapid rotators~\citep{mcquillan2013stellar} seem close to low-integer spin-orbit resonance. However, proximity to spin-orbit resonance is sensitive to the highest integer included in the calculation. If we choose the maximum low integer for m or n to be smaller than 10 the pattern in Fig.~\ref{fig:spinorbit} disappears.

%We therefore favor the observational bias explanation: Only fast rotating stars can be close to a low integer spin-orbit commensurability with a planet in a few days long orbit. Incidentally, only fast rotating stars can have strong enough magnetic fields to put their planets in the sub-Alfv\'enic zone. A comparison with the tidal interaction scenarios, where we do not expect to be biased towards fast rotating stars, further supports this conclusion: No clustering in spin-orbit commensurability appears for any tidal interaction scenario.


% \begin{figure}[t]
%     \script{paper_spin_orbit_commensurability.py}
%     \begin{centering}
%         \includegraphics[width=\linewidth]{figures/PAPER_spin_orbit_commensurable_p_spi_sb_bp1_norm.png}
%         \caption{
%            \textbf{Top panel}: Same as the bottom left panel in Fig.~\ref{fig:adtest_bp}, but color coded as below. \textbf{Bottom panel:} Spin-orbit commensurability compared to flaring star-planet interaction signal. The lower the p-value of the AD test, the stronger the interaction signal. Lower relative difference between the measured spin-orbit ratio and the integer ratio closest to it indicates a tighter spin-orbit commensurability. We include only ratios of integers $<10$. \textbf{Black diamonds:} All systems with p-values lower than that of K2-25, and expected power of interaction equal or higher than TOI-540's, the active 'SPI on' branch~(see top left panel in Fig.~\ref{fig:adtest_bp}). \textbf{Red crosses:} The complementary 'SPI off' branch includes all systems with expected power of interaction above that of TOI-540, but with p-values below that of K2-25. \textbf{Blue circles:} The remaining systems include all with expected power of interaction below that of TOI-540. Among the systems with high expected power, those with higher interaction signal ('SPI on') show closer spin-orbit commensurability than those on the inactive 'SPI off' branch.
%         }
%         \label{fig:spinorbit}
%     \end{centering}
% \end{figure}


\section{Summary and Conclusions}
\label{sec:summary}

We conducted the so far largest search for flaring star-planet interactions. Using over 7200 Quarters and Sectors of photometric monitoring from Kepler and TESS archives, we searched for flares in over 1800 systems of the about 3000 listed in the NASA Exoplanet Archive. In the single star systems, where flares could be unambiguously attributed to the planet host, and showed three or more energetic events in the data, we searched for flares that clustered in orbital phase. Our final sample consisted of 25 systems, among them well-known flaring hosts like Proxima Cen, AU Mic, and TRAPPIST-1. 

Applying both the stretch-and-break and Alfv\'en wing mechanisms for magnetic interaction, we found a tentative evidence trend between the expected power of interaction, and the presence of excess flaring in phase with the orbital period $P_{\rm orb}$. In particular, we found that there may be two branches at high expected power -- one where the measured excess flaring is correlated with the expected power, and one where excess flaring is absent. This may be an effect of the extent of the Alfv\'en surface, the intermittency in the interaction, or of the viewing geometry of the systems.

We also searched the same data for signs of locally expressed tidal interaction, that is, excess flaring in phase with the tidal bulges ($P_{\rm orb}/2$). We found a similar trend with different scaling laws of tidal interaction, albeit at even lower significance. Our system-by-system comparison suggests that young systems, like HIP 67522, may interact both tidally and magnetically, whereas old, magnetically inactive systems, like Kepler-42, might mostly interact tidally. 

Our study is the first that systematically searched for flaring star-planet interactions, and is not limited by sampling or selection bias in the way earlier studies of this effect have often been. While our results remain tentative, future studies using our technique will benefit from accumulating data. The TESS mission continues to produce high precision light curves for many nearby active planet hosts, and the PLATO~\citep{rauer2014plato} mission will join in late 2026. For AU Mic, we may be only a few TESS Sectors away from a $3\sigma$ level confirmation~\citep{ilin2022searching} of the tentative interaction signal observed here and in other work~\citep{klein2022one}. Several systems in our sample are already scheduled for further observing with TESS, including HIP 67522, Kepler-42, and KOI-12.

%Close-in planets are being explored as in-situ probes of the dynamic magnetic field~\citep{trigilio2023starplanet}, and stellar mass loss~\citep{vidotto2017exoplanets} conditions in their orbit. They are two key components of the space weather that a potentially habitable planet will encounter. The orbital distances, where sub-Alfv\'enic interaction can take place, are usually closer to the star than its habitable zone. However, in late M dwarfs like TRAPPIST-1, these two regions overlap~\citep{strugarek2021physics}, and information from closer in for earlier spectral types is still crucial to reliably model the magnetic environment of further out terrestrial planets. We only begin to uncover the invaluable information that close-in planets can provide about the ever-changing environment of habitable-zone terrestrial planets. 

\section*{Acknowledgements}
This paper includes data collected by the TESS mission, which are publicly available from the Mikulski Archive for Space Telescopes (MAST).
Funding for the TESS mission is provided by NASA’s Science Mission directorate. 
This work has made use of data from the European Space Agency (ESA) mission {\it Gaia} (\url{https://www.cosmos.esa.int/gaia}), processed by the {\it Gaia} Data Processing and Analysis Consortium (DPAC, \url{https://www.cosmos.esa.int/web/gaia/dpac/consortium}). Funding for the DPAC
has been provided by national institutions, in particular the institutions participating in the {\it Gaia} Multilateral Agreement.
%follow instructions here https://gea.esac.esa.int/archive/documentation/GEDR3/Miscellaneous/sec_credit_and_citation_instructions/
\section*{Data Availability}
All analysis script can be found under https://github.com/ekaterinailin/flaring-spi/. All data and scripts used to creat the tables and figures in this paper can be found under

\bibliography{bib}

\appendix
\section{Notes on individual stars}
\subsection{Proxima Cen}
Although Proxima Cen is part of a triple stellar system, we treat it a single star in this work because of the large angular separation of over 2 deg to its companions, $\alpha$ Cen AB.

\subsection{K2-354}
We note that we found K2-354 under EPIC 211314705 and K2-329(!) in \cite{bouma2020cluster}, who refer to the detection paper \cite{pope2016transiting}, and also under TIC 468989066.

\section{Expected power vs. AD test, color-coded by Rossby number}
\begin{figure*}[ht!]
    \script{paper_adtest_vs_value_scatterplots.py}
    \begin{centering}
        \includegraphics[width=\linewidth]{figures/PAPER_ADtest_Ro.png}
        \caption{
        Expected power of SPI vs. AD test results, assuming the same four different scenarios as in Fig.~\ref{fig:adtest_bp}, color-coded by Rossby number. $R\rm o=0.3$ is chosen to mark the transition from the saturated ($R\rm o < 0.3$) to the unsaturated ($R\rm o > 0.3$) activity regime. 
        }
        \label{fig:adtest_ro}
    \end{centering}
\end{figure*}

\end{document}



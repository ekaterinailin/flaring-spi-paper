% Define document class
\documentclass[twocolumn]{aastex631}
\usepackage{showyourwork}

% Begin!
\begin{document}

% Title
\title{Flaring star-planet interactions in star-planet systems in Kepler and TESS}

% Author list
\author{Ekaterina Ilin}
\author{Katja Poppenh\"ager}

% Abstract with filler text
\begin{abstract}
    The star-planet systems discovered so far are nothing like the Solar System. Many stars host planets in orbits much closer than Mercury, where it takes them mere days or even hours to complete a full revolution. In some of these systems, the planets are so close that they plow through the extended magnetosphere of the star, and are expected to trigger flares on their surfaces. So far, however, these flaring star-planet interactions escape our detection methods. If found, the planets and the flare they cause become localized probes of the parts of the stellar magnetosphere that extends from the location of one to the other. A measurement of these interactions constrains mass loss from the star, and potentially also the magnetic field of the planet. 

    In this work, we searched star-planet systems observed by Kepler and TESS for flaring star-planet interactions. We compiled a catalog of all flaring star-planet systems known to date, and characterized the detected flares detected. We then searched for flares in excess of the intrinsic flaring of the star, i.e., flares triggered by the interaction with the planet in phase with its orbit. 

    We find XXXX star-planet systems with > XXXX flares. For XXXX of them, we can quantify both the intensity of excess flaring, and the expected amplitude of interaction based on the system's magnetic field strength, orbital parameters, rotation and planet radius. Most systems are neither expected to show interactions, nor do they show any excess flaring. Among those systems that have high expected power of interaction, we see two branches. An inactive one, where no signs of excess flaring appear, and an active one %We find that the trend of higher interaction signal agrees with the expected strength, but also shows signs of intermittency, both for individual systems, and in comparison between ones with similar expected amplitude. HIP 67522, a 17 Myr young Sun with a Jupiter in a 9 day orbit, shows both one of the highest expected and the highest measured interaction amplitudes, albeit only at the $2.5\sigma$ level. 
\end{abstract}

% Main body with filler text
\section{Introduction}
\label{sec:intro}
One of the most surprising results in exoplanet research is the diversity of star-planet system architectures. In particular, the past two decades revealed that planets can occur in orbits so close to their host star that they could be considered to live inside the stellar atmosphere. At this proximity, the planet exerts force, gravitational and magnetic, both short- and long-term, on its host -- a phenomenon unknown to the Solar System. The individual processes are typically subsumed under the term star-planet interactions that emphasizes that they are bidirectional, affecting both planet, and star. 

Understanding short-term, magnetic star-planet interactions holds promise to provide important insights into the habitability of the entire system. Planets involved in such interactions can be viewed as in-situ probes of the system's space weather, and could prospectively reveal otherwise difficult to measure planetary properties such as the strength and extent of their magnetic fields.

Magnetic star-planet interactions can only occur if the planet is orbiting within the Alfv\'en radius of the star longer than it takes for the effects of the planet to travel to the surface of the star. The Alfv\'en radius is the radius at which the stellar wind velocity, which increases with increasing distance from the star, exceed the Alfv\'en velocity of the magnetized plasma. Beyond this radius, this plasma is disconnected from the star. Alfv\'en waves set off by the planet within the magnetic field they cross cannot propagate back to the star if the wind carries the plasma away faster than these waves can travel. In the sub-Alfv\'enic regime, the wave can reach the star, and deposit its energy in some way. \citet{lanza2012starplanet, lanza2018closeby} and \citet{zarka2007plasma, saur2013magnetic} suggest different mechanisms for when and how the energy dissipation takes place, and how much energy can be transferred and dissipated at all. 

One way of dissipating the energy in the magnetic field is through flares. 
Flares are explosions in stellar atmospheres that occur in the stellar corona as a way to dissipate energy stored in the magnetic field by the motion of footpoints of coronal loops. They are strong and impulsive eruptions released by reconnection and subsequent relaxation of field lines~\citep{svestka1976solar,priest2002magnetic}. We can observe flares in nearly all electromagnetic bands -- from radio to X-ray~\citep{benz2010physical}. In contrast to solar flares, stellar flares often are extremely energetic, sometimes enhancing the stellar flux by orders of magnitude~\citep{maehara2012superflares, shibayama2013superflares, paudel2018k2}. But since they occur randomly in time, individual events can be hard to catch.

In the 2010s, a wealth of flare observations from space missions like Kepler~\citep{borucki2010kepler} and K2~\citep{howell2014k2} have been catalogued~\citep{davenport2016kepler, paudel2018k2, ilin2021flares}. We now have access to long-term stellar monitoring data suitable for flare hunting. Currently, the Transiting Exoplanet Survey Satellite~\citep{ricker2015transiting} is rapidly expanding the treasury~\citep{gunther2020stellar}, which will grow even further after the launch of the PLATO mission in 2026~\citep{rauer2014plato}. As will become clear in the following paragraphs, the key advantage of these missions is their quasi uninterrupted monitoring of star-planet systems that covers all phases of their planets' orbits.

The absolute power of flaring star-planet interactions is difficult to constrain, so we do not know how energetic planet-induced flares can be. The estimates of the maximum dissipated power~\citep{lanza2018closeby} are well within the regime of superflares typically observed by Kepler and TESS. However, on the one hand that power is an upper limit, so the observed energies might be lower. The planet might for instance be triggering flares prematurely, leading an overabundance of low energy flares~\citep{loyd2023flares}. On the other hand, the energy need not be dissipated continuously, particularly in the stretch-and-break mechanism, so it might be higher. We also do not know whether flares triggered by interactions with a companion would have different properties than intrinsic flares. We assume that, since they would be caused by magnetic reconnection triggered by the planet in the stellar corona, they should follow the same process as intrinsic flares. 

Since planet-induced flares are not expected to look any different from intrinsic flares, and we do not know how their energies will be distributed, we require a method to distinguish them without referring to their individual properties. Instead, we can look at the statistical distribution of flares in time: 

If we picture the magnetic field lines connecting the planet to the star in the sub-Alfv\'enic zone, we expect that the stellar footpoint of these lines will move along with the planet. Flares triggered by the planet will occur at a preferred orbital phase, namely whenever the footpoint is visible to the observer. Depending on the system's architecture and magnetic situation, i.e., the inclination of the rotation axis, the spin-orbit (mis-)alignment, and the extent and configuration of the large scale magnetic field -- the footpoint could be always in view, or never. In expectation, however, in star-planet systems with favorable architectures and magnetic field states, the footpoint visibility will be modulated in phase with the planetary orbit. The resulting phase preference of planet-induced flares introduces a deviation from a random distribution of flares. Space missions like Kepler and TESS provide light curves that cover all phases of the planet's orbit, so that phase-correlated planet-induced flares can be detected against a background of uncorrelated intrinsic flaring. 

\citet{fischer2019timevariable} and \citet{howard2021evryflare}

In this work, we searched for excess flaring in all available Kepler and TESS short cadence light curves of confirmed star-planet systems~(Section~\ref{sec:data}). We scoured the light curves for flares, and tested their observed phases for departures from a uniform distribution in phase with the innermost planet's orbit~(Section~\ref{sec:methods}). We present the catalog of all flares found in our sample of star-planet systems, and compare the test results with the expected power of interaction for each system~(Section~\ref{sec:results}). We discuss the results for the most interesting star-planet systems individually~(Section~\ref{sec:results:individualstars}, and take a look at the more general patterns emerging from our systematic analysis~(Section~\ref{sec:discussion}). Finally, we summarize our findings, and suggest future observations that could uncover the star-planet interaction mechanisms operating behind the observed trends~(Section~\ref{sec:summary}). 




%Short intro to SaB mechanism

%Short intro to AW mechanism

% ----------------------------------------------------------------------------------

% In both models, the radius of the perturbing body, and its proximity increase the power of interaction. Would it not be more reasonable to search for star-brown dwarf or star-star interactions? 

% Flares caused by magnetic interactions in close binary systems were previously observed. Flares detected on the young eccentric binary system V 773 Tau could be attributed to colliding magnetospheres~\citep{massi2002periodic, massi2008interacting}, while other promising systems, like DQ Tau, another young equal-mass binary system in a strongly eccentric orbit, did not live up to the expectations~\citep{salter2008captured,salter2010recurring,getman2011young,getman2016search, kospal2018spots}. In these systems, it is, however, unclear whether it is mass transfer or reconnection that drive excess coronal emission and flares at periastron. Substellar objects can orbit the star much closer than a stellar companion without filling the Roche lobe. Gravitationally, a rocky planet can go unnoticed by its host even if it orbits at ultrashort periods of less than a day. 
% Interactions in systems with brown dwarf companions are less likely than binary systems 



\section{Data}
\label{sec:data}
In this study, we wanted to compile the largest possible sample to search for flaring star-planet interactions, so took the entire Planetary Systems Composite Parameters Table (PSCP, Section~\ref{sec:data:sps})\footnote{ \url{https://exoplanetarchive.ipac.caltech.edu/cgi-bin/TblView/nph-tblView?app=ExoTbls&config=PSCompPars}} from the NASA Exoplanet Archive, and scanned the systems for available Kepler and TESS light curves~(Section~\ref{sec:data:photometry}). We ended up searching a total of 1812 systems and over 7200 light curves from both missions for flares. To calculate the amount of excess flaring from star-planet interaction, we then required the orbital periods of each system~(Section~\ref{sec:data:orbitalperiod}). To compare our measurements to theory, we also wanted to estimate the expected power of interaction for each system with at least three flares. For this, we required further system parameters -- distance between planet and star, stellar rotation periods, planetary radii, and stellar luminosity~(Sections~\ref{sec:data:orbitalperiod}-\ref{sec:data:lum}). We started compiling the table of system parameters with the PSCP, but verified, supplemented and updated the table from the literature. The literature parameters for the systems in the final sample of systems, for which we performed the full analysis, is shown in Table~\ref{tab:maintable_lit}.

\subsection{Star-planet systems}
\label{sec:data:sps}
We compiled our sample of star-planet systems from the PSCP, from which we removed controversial planet detections (``pl\_controv\_flag'' set to 1), i.e. detections with existing literature questioning the result. The catalog then contained 2993 transiting and 191 non-transiting systems, from which we picked the innermost known planet in each system. 

\subsection{Kepler and TESS photometry}
\label{sec:data:photometry}
The Kepler and TESS missions are unbeatable with respect to long-term optical monitoring of stellar flares. Their excellent coverage of orbital phases makes the light curves ideal for our search for orbital phase dependent flare timing.

Between 2009 and 2013, the Kepler space telescope~\citep{koch2010kepler} nearly continuously observed a patch of the sky in the Cygnus-Lyra region. Each of the 18 observing Quarters contains nearly uninterrupted $\sim 90$ days of observations, totaling over 100000 stars monitored in 2-min cadence. 
% 100000 from here: https://exoplanetarchive.ipac.caltech.edu/docs/KeplerMission.html

The Transiting Exoplanet Survey Satellite~(TESS,~\cite{ricker2015transiting}) is an all-sky mission that began operations in 2018, completed two sky scans until summer 2022. It is observing at the time of writing, collecting nearly continuous photometric time series in the 600-1000 nm band for $\sim 27 days$ in each observing Sector. About $200\,000$ stars have been observed in 2-min cadence in the first two years of operations with about 20000 targets per Sector. Out of these, from Sector 27 on, 1000 targets were observed at even higher 20-s cadence in each Sector. 
% 20000 per sector does not add up, because of stars observed in multiple sectors, the 20000 and 1000 figures can be found here: https://tess.mit.edu/observations/target-lists/
% extended TESS mission https://heasarc.gsfc.nasa.gov/docs/tess/extended.html

Based on the filtered PSCP table, we queried the full Kepler archive (quarters Q0-Q17, Data Release 25), and the most recent TESS catalog (July 2022) for their respective 1-min and 2-min/20-s cadence light curves. In total, we searched 7258 light curves for flares -- 3030 Kepler Quarters, and 4228 TESS Sectors. We found and analyzed light curves for a total of 1811 systems, 344 of which were observed only by the primary Kepler mission, 1205 only by TESS, and 262 by both missions. We did not use K2 light curves because of the many systematics, and consequently the high costs of de-trending them. But we included the flare list obtained by~\citet{paudel2018k2} for TRAPPIST-1 K2 flares, since this system is a potential candidate for sub-Alfv\'enic interactions~\citep{fischer2019timevariable}.

\subsection{Orbital periods and transit mid-times}
\label{sec:data:orbitalperiod}
We adopt the orbital periods from the PSCP, which were either obtained from Kepler or TESS transits, or from radial velocity measurements. For the transiting systems, we use the transit mid-times given in the PSCP to set orbital phase zero. If possible, we do this for each mission separately to reduce the uncertainties on orbital phase. For the non-transiting system, phase zero is set arbitrarily.
\subsection{Semi-major axes and eccentricities}
\label{sec:data:a}
The critical parameter is the distance between the star and the planet, not the semi-major axis itself. This varies if the orbit is eccentric. To account for varying distance, we adopt the mean semi-major axes from the literature, but use a custom estimate for the uncertainty that includes the eccentricity.

If the eccentricity is known, then the uncertainty on the distance is set to either the error on the semi-major axis, or to half of the difference between periastron and apoastron -- whichever is larger. If eccentricity is not known, then the uncertainty on the distance is either the 25\% error on the semi-major axis, that is, assuming $e=0.25$, or the uncertainty on the semi-major axis -- whichever is larger. We chose $e=0.25$ because it is both a typical value within our sample~(see Table~\ref{tab:maintable_lit}), and in the literature~\citep{eylen2019orbital}.

\subsection{Rotation periods}
\label{sec:data:rotationperiods}
We adopt available rotation period values from the literature. Almost all stem from light curve variability~\citep{angus2018inferring, mazeh2015photometric, mcquillan2013stellar, mcquillan2014rotation, luger2017sevenplanet, stock2020carmenes, deleon202137, torres2017validation, stefansson2020habitable, zicher2022one, ment2021toi, rizzuto2020tess}, and only one from periodic variation in chromospheric lines~\citep{demangeon2021warm}. When uncertainties are not given, we conservatively assume $10\%$ uncertainty, WHERE ARE THE PERIODS FROM.


\subsection{Planetary radii}
\label{sec:data:planetradii}
For the transiting planets, we adopt the literature values for planet radius $R_p$, and uncertainties, from the PSCP. For the non-transiting radial velocity detected planets we use the planetary mass $M_p$ or $M_p\sin i$ to calculate the radius or a lower limit for the radius using the empirical relations derived in \cite{chen2017probabilistic} using their open source \texttt{forecaster} tool, upgraded to \texttt{astro-forecaster}\footnote{https://pypi.org/project/astro-forecaster/} by Ben Cassesse. We note that for GJ 674, the mass estimates in \cite{bonfils2007harps} and \cite{boisse2011disentangling} do not quote uncertainties, but differ by 0.3MEarth, so we assumed that value as the uncertainty on $M_p\sin i$.

\subsection{Bolometric luminosity}
\label{sec:data:lum}

We take bolometric luminosity values as given in the PSCP whenever they are given with uncertainties. We supplement missing values, and replace entries without quoted uncertainties with Gaia DR3 FLAME~\citep{fouesneau2022gaia} solutions~(\texttt{lum\_flame}). 

\section{Methods}
\label{sec:methods}
We measure flaring star-planet interactions (SPI) as the presence of excess flares triggered by the orbiting planet. In the absence of flaring SPI, flare peak times will be distributed randomly in orbital phase. In the presence of flaring SPI, we measure a phase dependent deviation from this randomness.

The main data for this analysis are flare times. To obtain them, we gather the Kepler and TESS light curves for all star-planet systems, remove rotational variability trends, search the de-trended light curves for flares, and convert the flare times to orbital phases. We then perform a customized Anderson-Darling test on the flare peak time distribution, which yields a p-value for the significance of the SPI signal. Finally, we compare these results to the theoretically expected SPI power $P_{SPI}$ in each system.

The methods for light curve de-trending and flare finding in Kepler and TESS light curves, as well as the Anderson-Darling test, are the same as detailed in~\citet{ilin2022searching}. We briefly recap the techniques in Sections~\ref{sec:methods:flaresearch} and \ref{sec:methods:adtest}. The expected power of SPI depends on the relative velocity between the planet and the magnetic field strength in its orbit, the derivation of which we explain in 
Sections~\ref{sec:methods:relvel} and \ref{sec:methods:bfield}, respectively. With relative velocity and magnetic field strength derived, we can combine them with stellar radius, planetary radius, and semi-major axis to estimate the power of SPI. We use the scaling laws from two different magnetic SPI models, which we introduce in Section~\ref{sec:methods:pspi}.

\subsection{Light curve de-trending and flare search}
\label{sec:methods:flaresearch}
To remove trends and rotational variability from the light curve without losing the flare signal, we use an empirically derived multistep process. First, we apply spline fit with a coarse sampling of 30h mean values to capture slow rotation with periods above multiple days and non-periodic trends. Then, we iteratively fit a series of sines to capture rotational signal on time scales down to about half a day, close to the fastest rotational signals measured in low mass stars. Eventually, we apply two Savitzky-Golay~\cite{savitzky1964smoothing} filters in sequence, with window sizes of 6h and 3h each. At this stage, we mask all data points above a $2.5 \sigma$ (or $1.5 \sigma$ for very active stars like AU Mic or Proxima Cen) threshold as flare candidates to prevent the filter from ironing out the flares. As a final step, we fit exponential functions to the edges of the light curves, if there are data points that deviate more than one standard deviation from the median value, while keeping the flare candidates masked.

In these light curves, we then search for flares as series of at least three consecutive data points $3\sigma$ above the noise level, implemented in AltaiPony~\cite{ilin2021altaipony}. We estimate the noise
level as the standard deviation in a rolling window of two hours, while masking deviation above $2.5\sigma$ (or $1.5$ sigma for active stars). To capture the exponential tail of the flare, we use the \texttt{addtail} flag to continue adding data points to the end of the flare as long as they are $2\sigma$ above the noise threshold. For these flares, we calculate the relative amplitude $a$ and equivalent duration $ED$, defined as the integrated flare flux $F_{flare}$ over the duration of the flare, divided by the median flux $F_0$ of the star, integrated over the flare duration~\citep{gershberg1972results}:
\begin{equation}
\label{eq:ED}
ED=\displaystyle \int \mathrm dt\, \frac{F_{flare}(t)}{F_0}.
\end{equation}
Equivalent duration is the time during which the non-flaring star releases as much energy as the flare.


\subsection{Custom Anderson-Darling test}
\label{sec:methods:adtest}
For each star-planet system with three or more flares in their Kepler and/or TESS light curves, we test for deviations from a random distribution of flares with orbital phase. We use the same customized Anderson-Darling test as in \cite{ilin2022searching}. In brief, we first take the number of flares observed in the TESS and Kepler data, calculate a base flare rate for each light curve. We then calculate how often each orbital phase has been covered by the data, again for each light curve. With the phase coverage and base flare rate combined, we can tell how many flares we would expect to see in any given phase bin if flares were randomly distributed. This number of expected flares per bin can be aggregated into one expected distribution by summing over all available light curves. As a last step, this distribution is then compared to the observed one using an Anderson-Darling test. We repeat this test with different phase offsets to account for potential biases in sensitivity of the test at different phases.

The only adjustment to the procedure in \cite{ilin2022searching} is that we use four equidistant start phases (i.e., 0, 0.25, 0.5 and 0.75) compared to the 20 used in~\cite{ilin2022searching} to save time, and because the range of derived $p$-values is already well-sampled using four. We adopt the standard deviation of these four $p$-values as the uncertainty on the flaring SPI measurement.



\subsection{Relative velocity}
\label{sec:methods:relvel}
We calculate the relative velocity between the magnetic field of the star, which we assume is co-rotating with the stellar rotation period $P_{rot}$, at the planetary orbit, where the planet moves with a period $P_{orb}$:

\begin{equation}
    v_{rel} = 2 \pi a \left(\frac{1}{P_{orb}} - \frac{1}{P_{rot}}\right).
\end{equation}

We use quadratic error propagation to estimate the uncertainty in $v_{rel}$ with the mean of the upper and lower uncertainty values on the orbital period and rotation period, if both are given, and the uncertainty on semi-major axis $a$ as derived in Sec.~\ref{sec:data:a}. 

\subsection{Stellar magnetic fields}
\label{sec:methods:bfield}
We derive the average magnetic field strength $B$ from $Ro$ using the empirical relation derived in \cite{reiners2022magnetism},  Table 2, in the unsaturated and saturated regimes, respectively:

\begin{eqnarray}
    B &= 199\,\text{G} \cdot Ro^{-1.26\pm 0.1} \;(\text{if}\; Ro > 0.13) \\
    B &= 2050\,\text{G} \cdot Ro^{-0.11\pm 0.03} \;(\text{if}\; Ro < 0.13) 
\end{eqnarray}

The estimate is consistent with existing Zeeman broadening for, e.g. AU Mic (3010 G~\cite{reiners2022magnetism}).
\subsection{Power of star-planet interaction}
\label{sec:methods:pspi}
We consider two theories for the mechanism behind flaring star-planet interactions, the stretch-and-break and the Alfv\'en wing mechanisms. For both of them, we include the case of a magnetized and an unmagnetized planet, for a total of four estimates of the expected power of magnetic star-planet interaction.

First, we estimate the power of star-planet interaction generated by the stretch-and-break mechanism using the scaling relations in~\cite{lanza2012starplanet} (their Eqn. 45), where $P_{sb}$ is proportional to

\begin{equation}
    P_{sb} \sim B_p^{2/3} B_*^{4/3} v_{rel} R_p^2 a^{-4} R_*^4.
\end{equation}

$B_p$ and $B_*$ are the total planetary and stellar surface field strengths, respectively; $v_{rel}$ is the relative velocity between the stellar rotation and the planet's orbit at the semi-major axis; and $R_*$ and $R_p$ are the stellar and planetary radii, respectively; and $a$ is the semi-major axis.

Second, we also consider the case of an unmagnetized planet with $B_p=0$, in which case Eq.~45 in~\cite{lanza2012starplanet} reduces to

\begin{equation}
    P_{sb0} \sim B_*^2 v_{rel}  R_p^2  a^{-4} R_*^4
\end{equation}

Third, for comparison, we estimate the power of star-planet interaction generated by the Alfv\'en wing mechanism ~\cite{saur2013magnetic,kavanagh2022radio}, combining Eqs.~8~and~11 in \cite{kavanagh2022radio} to give

\begin{equation}
    P_{aw} \sim R_{p}^2  B_p^{2/3}  B^{1/3}  v_{rel}^2 a^{-2}  
\end{equation}
where we approximate the stellar wind density $\rho_W$ with $a^{-2}$ and stellar wind magnetic field at the planetary radius $B_W$ with the surface field strength $B a^{-3}$, which result in the total $a^{-2}$ factor.

And fourth, we finally consider the case of an unmagnetized planet with $B_p=0$, in which case Eq.~8 in \cite{kavanagh2022radio} reduces to

\begin{equation}
    P_{aw0} \sim R_p^2  B  v_{rel}^2  a^{-4}
\end{equation}


\section{Results}
\label{sec:results}

Our goal was to measure magnetic star-planet interaction as the presence of excess flares in phase with the innermost planet's orbit. We searched all star-planet systems that were observed with Kepler and TESS at 1 or 2 min cadence, respectively, for flares. For the star-planet systems in the resulting flare catalog~(Section~\ref{sec:results:catalog}), we confirmed that the orbital phase could be known well enough for the entire observing baseline of the system~(Section~\ref{sec:results:coherence}). Knowing that, we could move on to calculate the orbital phases of each flare, and calculate distributions for each system in the catalog~(Section \ref{sec:results:phasedist}). We then applied the custom Anderson-Darling test to check how likely it is that some of the flares were triggered by the planet. Comparing the test results with the expected power of magnetic star-planet interaction, we found that the probability of observing planet induced flares increases with the expected power in all considered scenarios~(Section~\ref{sec:results:spi}). However, not all systems with high expected power show signs of flaring SPI, which appears as a branching in the distribution of test results, and could be due to the intermittency of the interaction. Finally, we provide additional context for the most interesting systems in our sample, and explain why we had to exclude some systems from the analysis in Section~\ref{sec:results:individualstars}.
\subsection{Flare catalog}
\label{sec:results:catalog}
We searched a total of 3032 Kepler Quarters and 4181 TESS Sectors of a total of 1811 star-planet systems for flares. We inspected all candidates manually, and added flares observed by K2 on TRAPPIST-1~\cite{paudel2018k2}, and flares from the planet hosting primary in the Kepler-411 binary~\cite{jackman2021stellara}. The final table contains a total of \input{output/PAPER_total_number_of_flares.txt} flares in \input{output/PAPER_total_number_of_systems_with_flares.txt} systems. In Table~\ref{tab:flares}, we list all the flares and their characteristics. To identify each event, we provide the name of the system in which the flare was found, along with its TIC, whether it was observed with Kepler or TESS, and during what Quarter or Sector, respectively. For each flare, we also give the start and finish time, relative amplitude $a$ and equivalent duration $ED$. If the transit midtime is known for the innermost planet, we also give the orbital phase at which the flare occurred. The full table is provided with the supplementary material and via Zenodo (see Data Availability).

\begin{table*}
    \script{paper_latex_flare_table.py}
    \centering
            \caption{
            Flare catalog of all star-planet systems observed by Kepler and TESS. In transiting multi-planet systems, the orbital phase refers to the innermost planet, with the transit mid-time at phase zero. 
        }
    \input{output/flare_table.tex}
        \label{tab:flares}
\end{table*}

\subsection{Period coherence times}
\label{sec:results:coherence}
In many cases, the observing baseline covered by Kepler and TESS for a given system can span multiple years. In these cases, the orbital period of the innermost planet must be known very precisely so that we can assign accurate orbital phases to the flare events.
We test this by dividing the timespan between the first and the last flare in the combined Kepler and TESS observations by the coherence time $\tau$ of the orbital period for each system. We calculate the coherence time as

\begin{equation}
    \tau = P^2 / \sigma_P,
\end{equation}
where $P$ is the orbital period, and $\sigma_P$ is its uncertainty.
The resulting ratio should be $<<1$. The bottom panel in Figure~\ref{fig:coherence_hist} shows a histogram of these ratios in our sample. The orbital period is known sufficiently well for our analysis in most cases with ratios $<0.02$. The three cases where the ratio is highest with $0.05$, $0.12$, $0.24$ deviation are TAP 26, GJ 3082, and GJ 674, respectively. We keep these systems in our sample, but caution that the measured absence of flaring SPI might be due to the uncertain orbital period of the planet in these cases.

We repeated the coherence calculation for $P_{rot}$, and found that the ratio is $>.1$ for most stars~(Figure~\ref{fig:coherence_hist}, top panel). There are three reasons for this: First, the uncertainty for the rotation was not always provided. Second, rotation period is more difficult to ascertain from a light curves' variability than the orbital period, particularly if the modulation is weak and or rapidly evolving. And third, differential rotation causes spots at different latitudes to move at different speeds, so it becomes difficult to even define a rotation period, as long as we don't know at which latitude(s) the flares occur.

\begin{figure}[ht!]
    \script{paper_coherence_histogram.py}
    \begin{centering}
        \includegraphics[width=\linewidth]{figures/PAPER_coherence_histogram.png}
        \caption{
           Time span of observation vs. coherence time of the orbital and rotational periods, respectively. Orbital periods are all known precisely enough, so that the phase uncertainty at the last observed flare is of the order of $10^{-2}$. In contrast, in stars with known rotation periods, the precision is often too low, so that the phase of the last flare is often undetermined (ratio on x-axis $\sim 1$).
        }
        \label{fig:coherence_hist}
    \end{centering}
\end{figure}

\subsection{Flare phase distributions}
\label{sec:results:phasedist}
The Anderson-Darling test we use compares the measured distribution of orbital phases of the flares to the expected distribution, and return the significance of the deviation between the two. In the expected distribution, the same number of flares would be distributed randomly across the observing time. Figures~\ref{fig:cumdist_transiting} and \ref{fig:cumdist_rv} show the flare phase distributions along with the corresponding expected distributions for transiting and non-transiting systems, respectively. The expected distributions usually deviate from a straight line because we take into account the coverage of the orbital phases by the Kepler and TESS observations, as well as the different detection thresholds in each light curve.

\begin{figure*}[ht!]
    \script{paper_cumdist_individual_transiting.py}
    \begin{centering}
        \includegraphics[width=.95\linewidth]{figures/PAPER_flares_phase_hist_transiting.png}
        \caption{
            Cumulative distributions of orbital phases of flares in the transiting planet hosts observed by Kepler and TESS. The bisector line is dotted, the expected distribution is solid blue, and the observed distribution is solid black. Phase zero corresponds to the transit mid-time of the planet. 
        }
        \label{fig:cumdist_transiting}
    \end{centering}
\end{figure*}

\begin{figure*}[ht!]
    \script{paper_cumdist_individual_rv.py}
    \begin{centering}
        \includegraphics[width=0.95\linewidth]{figures/PAPER_flares_phase_hist_rv.png}
        \caption{
            Cumulative distributions of orbital phases of flares in the RV planet hosts observed by Kepler and TESS. The bisector line is dotted, the expected distribution is solid blue, and the observed distribution is solid black. Phase zero is chosen arbitrarily. 
        }
        \label{fig:cumdist_rv}
    \end{centering}
\end{figure*}

\subsection{Flaring star-planet interaction signal}
\label{sec:results:spi}
We used the custom Anderson-Darling test introduced in  \citet{ilin2022searching} and Section~\ref{sec:methods:adtest} to look for flaring SPI in all systems with three or more flares with equivalent duration $ED>1\,$s, detected in their Kepler and TESS observations. We choose the $ED$ cutoff to make sure that we are comparing similar flares regardless of spectral type, that is, flares with the same energy relative to stellar luminosity. We show the results that include all flares regardless of $ED$ in the appendix (XXXX TBD).

Table~\ref{tab:maintable_der} lists the $p$-value results for each star-planet system. We do not find any $>3\sigma$ signal of star-planet interaction in our sample. However, overall, the significance of the deviation increases with higher expected power of interaction. 

Figure~\ref{fig:adtest_bp} shows that in every scenario we considered (Alfv\'en wing or stretch-and-break mechanism, with or without planetary magnetic field), there are two branches. On one branch, the significance of the measured interaction is correlated with the expected power $P$. On the other branch, no interaction is measured regardless of the expected power $P$. 

Table~\ref{tab:maintable_der} lists the values for the four different powers $P$, calculated with the scaling laws introduced in Section~\ref{sec:methods:pspi} and normalized to AU Mic, along with the derived parameters for these scaling laws, i.e., Rossby number R$_o$, surface-average magnetic field strength $B$, and relative velocity between the planet and a co-rotating magnetic field $v_{rel}$. 


\begin{table*}
\movetableright=-20mm
\footnotesize
    \script{paper_main_table.py}
    \caption{Flaring single star-planet system parameters.}
    \input{output/table_lit_vals.tex}
        \label{tab:maintable_lit}
    \tablerefs{\input{output/lit_table_bibstring.tex}}
\end{table*}


\begin{table*}
\footnotesize
\movetableright=-20mm
\script{paper_main_table.py}
\caption{Flaring star-planet interaction. $Ro$, $B$, and $v_{rel}$, are derived from literature values (Table~\ref{tab:maintable_lit}). $P_{xx}$ stands for the power of stretch-and-break ($sb$) and ALfv\'en wing ($aw$) interaction mechanisms, assuming the planet has a magnetic field strength of 1 G. $P_{xx0}$ is the same, but assuming an unmagnetized planet. All powers are normalized to AU Mic. The $p$-value of the Anderson-Darling test is lower when the system shows more flares periodic with the planetary orbit.}
\input{output/table_der_vals.tex}
    \label{tab:maintable_der}

\end{table*}


\begin{figure*}[ht!]
    \script{paper_adtest_vs_value_scatterplots.py}
    \begin{centering}
        \includegraphics[width=\linewidth]{figures/PAPER_ADtest_bg.png}
        \caption{
            Expected power of SPI vs. AD test results, assuming four different scenarios, color-coded by stellar surface field strength.
        }
        \label{fig:adtest_bp}
    \end{centering}
\end{figure*}



\subsection{Individual stars}
\label{sec:results:individualstars}
The properties of star-planet systems in this work are diverse, including both very young and old systems (with slowly rotating host stars); systems with super-Earths, Neptunes, and Hot Jupiters; and spectral types covering the lower main sequence from early G to late M dwarfs. 

For our analysis, it is important to know that the observed flares indeed occurred on the planet host star. We therefore excluded some systems due to documented contamination from nearby (bound or co-moving) companions~(Section~\ref{sec:results:individualstars:excluded}). One exception to this rule is Kepler-411, a binary system for which the flare contributions from each component could be separated~(Section~\ref{sec:results:individualstars:kep411}). We also dropped GJ 1061 from the sample because its rotation period was unconstrained. Another uncertainty in our analysis is the possibility of additional planets at even shorter orbits than the currently known innermost planet, such as might be the case for Proxima Cen~(Section~\ref{sec:results:individualstars:proxima}). 

The bulk of the systems in our sample are neither expected to show high power of magnetic star-planet interaction, nor do they show any deviation from intrinsic stellar flaring that could be indicative of flaring star-planet interaction. In the subsample of stars with high expected power, of the order of AU Mic's and above, we first take a closer look at those that seem to follow an increase in measured flaring star-planet interaction with increasing expected power $P$, that is AU Mic itself, K2-25, TOI-540, and HIP 67522~(Sections \ref{sec:results:individualstars:aumic} and \ref{sec:results:individualstars:hip67522}). Then we consider those systems where high power was expected in all scenarios, but not measured, i.e., TAP 26 and KOI-12~(Sections \ref{sec:results:individualstars:tap26} and \ref{sec:results:individualstars:koi12}).
\subsubsection{Excluded targets}
\label{sec:results:individualstars:excluded}

While they appear in Table~\ref{tab:flares}, we excluded a number of systems from further analysis. We drop all multiple stars except for Proxima Cen (whose components are well separated both physically and on the sky), and all stars with nearby objects that contaminate our analysis, e.g., from ~\citet{ziegler2018measuring}: \input{output/multiples_string.tex}

For instance, Kepler-808 (KOI 1300) has a $\sim0.4M_\odot$ \citep{kraus2016impact} companion that is about 1.8 mag fainter than the primary at a separation of .78 arcsec~\citep{baranec2016roboao}. TOI-837 has co-moving M dwarf a few arcsec away in the same young cluster, 5 mag fainter, but the primary is an F9-G0 star, so flares could still originate come from both stars~\citep{bouma2020cluster}. HD 41004 B, DS Tuc A, and LTT 1445 are known multiple systems where all components contribute significantly to the total flux.

The only single star system excluded is GJ 1061. The star is lacking a stellar rotation period estimate, which we need to calculate the magnetic field, and $v_{rel}$. \citet{dreizler2020reddots} estimate that it is a fairly slow rotator with a period between 50 and 200 days, so we do not expect it to have a strong magnetic field anyway.

\subsubsection{Kepler-411}
\label{sec:results:individualstars:kep411}
Kepler-411 has 3 mag fainter companion~\cite{wang2014influence,ziegler2018measuring}, but we do not exclude it from the analysis. The system was observed in Kepler short cadence by \citet{jackman2021stellara} and both the planet host, and the companion star flare. \citet{jackman2021stellara} disentangle the contributions from each component on a pixel level. According to \citet{morton2016false, sun2019kepler411}, Kepler-411's planets orbit the primary companion. The fainter companion of Kepler-411 appears to cause the majority of flares (41), whereas the planet host causes only 7. Adopting the 7 flares from \cite{jackman2021stellara} in our analysis, we find that they all have equivalent duration $<1\,$s, which excludes Kepler-411 from further analysis.

\subsubsection{Proxima Cen}
\label{sec:results:individualstars:proxima}
We do not detect a deviation from intrinsic flaring on Proxima Cen, consistent with numerical models that place Proxima Cen b well outside the sub-Alfv\'enic zone~\citep{kavanagh2021planetinduced}. However, this might not entirely exclude Proxima Cen from the search for flaring star-planet interaction. The tentatively detected Proxima Cen d, a planet further in, at 0.029 au, or 5 day orbital period~\citep{faria2022candidate, artigau2022linebyline} could still cause flaring star-planet interactions. However, the orbital period of the tentative planet is so uncertain at this point that its coherence time is shorter than the roughly 2yr observing baseline of Proxima Cen, preventing us from accurately measuring the flare phases. 


\subsubsection{AU Mic, K2-25, and TOI-540}
\label{sec:results:individualstars:aumic}

AU Mic is a 16-29 Myr old pre-main sequence M0-M1 dwarf with a strong magnetic field of about $3010\pm220\,$G obtained from Zeeman broadening measurements~\citep{reiners2022magnetism}. The innermost Neptune-sized planet, AU Mic b, could be both inside the sub-Alfv\'enic zone if the star's mass loss rate is relatively low at about 30 times the solar value or lower. If the mass loss rate is high, exceeding the solar value by a factor of several hundreds, AU Mic b could orbit in the super-Alfv\'enic zone, and become unable to experience any planet-induced flaring. Recently, \citet{klein2022one} found tentative periodicity with the orbital period of AU Mic b in the chromospheric He I D line, which increases if the contribution from flares is included in the calculation. Overall, our marginal signal of flaring star-planet interaction in the AU Mic system supports the idea that this young, magnetically active M dwarf system with a close-in Neptune could exhibit planet=induced flaring.
For a detailed discussion of AU Mic and its flaring star-planet interaction, we refer to~\cite{ilin2022searching}, where we also estimate that an additional 50–100 days of TESS-like monitoring of AU Mic would yield a $3\sigma$ detection if the marginal signal in our data is real.
 
K2-25 is a multiplanet system around a fast rotating ($<2\,$d) mid-M dwarf in the Hyades open cluster, that is, 600-800 Myr old~\citep{stefansson2020habitable}. Both the expected power of interaction, and the measured deviation, are remarkably similar to the AU Mic system. Moreover, K2-25 b is in a presumably similar environment to GJ 436 b, but does not show the atmospheric escape in Ly$\alpha$ that the latter famously exhibits~\citep{rockcliffe2021lya}. This might at least in part be due to the stellar wind conditions around K2-25 b. Further investigation of magnetic star-planet interactions, flaring or otherwise, in this system could add an important constraint on K2-25's wind properties by clarifying whether the planet orbits in- or outside the Alfv\'en radius.

K2-25 is eccentric, see Eike Günther?

In Figure~\ref{fig:adtest_bp}, TOI-540 clusters with K2-25 and AU Mic -- a fast rotating M dwarf with its innermost planet in a $1.24\,$d orbit~\citep{ment2021toi}. Compared to the other two systems, its innermost planet is not a Neptune, but a rocky planet slightly smaller than Earth~\citep{ment2021toi}. It is a relatively smaller obstacle in the host's magnetic field, yet the interaction is expected to be of a similar magnitude due to the strongest inferred magnetic field and closest orbit among the three systems. 

In the scenario wherein the stretch-and-break mechanism drives the interaction, the three system cluster together most distinctly. In this scenario, all systems that do not show elevated interaction, have a weaker magnetic field below $2000\,$G~(top row in Figure\ref{fig:adtest_bp}). In the Alfv\'en wing scenario~(bottom row in Figure\ref{fig:adtest_bp}), the distinction is less clear, and systems with strong fields and similar expected power of interaction do not show any excess flaring. In both cases, the absence of interaction could be due to the Alfv\'en radius being within the planetary orbit (note that the scaling laws assume that the planet is sub-Alfv\'enic, but make no statement about it). All else equal, this is more likely if the stellar magnetic field strength is weaker. Another explanation is the possible intermittency, or 'on-off nature' of the interaction~\citep{shkolnik2008nature}, which may be at play here, but particularly in cases where the expected power is very strong, such as in the comparison between HIP 67522 and TAP 26 that we address in the following Section.

\subsubsection{HIP 67522}
\label{sec:results:individualstars:hip67522}
HIP 67522 has one of the strongest expected SPI signals in our sample, and shows the clearest sign of flaring SPI at $>2\sigma$ level, albeit with only 6 flares in the sample. These flares are distributed across two Sectors in TESS. In Sector 11, two flares occur at orbital phases $\sim 0.6$ and $\sim 0.8$. Two years later, in Sector 38, four flares occur, but this time, all of them take place at orbital phases $0.00-0.05$, that is, up to a few hours after transit.

HIP 67522 is a young Sun, currently contracting onto the main sequence. It is a Sco-Cen member (10-20 Myr old), which was discovered to host a close-in Jupiter, HIP 67522 b, in 2020~\citep{rizzuto2020tess}. Curiously, the planet is in close spin-orbit commensurability -- $P_{rot}/P_{orb}=5/1$, closer than any other system in our sample. Therefore, we cannot completely rule out that the observed periodicity is in fact a rotational periodicity, and not related to the planet. However, if we calculate the $p$-value of AD test using the rotational period instead of the orbital period, we find HIP 67522 consistent with uniform flaring in time within $1\sigma$. Fortunately, HIP 67522 will be observed again with TESS in April 2023, which may resolve the puzzle.

%It was found with low obliquity, consistent with XXXX~\citep{heitzmann2021obliquity}. Tentative third planet transiting.
%The system's $a/R_*\approx11.7$, eccentricity below .25~\cite{rizzuto2020tess}, so that  
 %\cite{wood2021characterizing} rule out stellar companions for HIP 67522 based on RV, high-resolution imaging, and Gaia data.

\subsubsection{TAP 26}
\label{sec:results:individualstars:tap26}
TAP 26 is a weak-line T-Tauri star with a strong magnetic field~\citep{yu2017hot}. \citet{lanza2018closeby} estimate that the system can release more energy in magnetic star-planet interaction than other  systems with Hot Jupiters, such as HD 179949, that has previously been detected with chromospheric variability in phase with the planet's orbit~\citep{shkolnik2008nature}. However, for TAP 26 b, the orbital period might is not well-constrained. \citet{yu2017hot} apply several methods to derive $P_{orb}$ from the radial velocity data: While the $10.8\,$d orbit we adopted is the most likely according to \citet{yu2017hot}, a $9.0\,$d orbit is also likely, and some of their applied methods favored a $13.4\,$d period. The absence of interaction could hence be a consequence of uncertain orbital period. However, intermittent interactions and the viewing geometry of TAP 26 could also explain the absence of phase-correlated flaring~(see Sections~\ref{sec:discussion:intermittency} and \ref{sec:discussion:viewing}).

\subsubsection{KOI-12}
\label{sec:results:individualstars:koi12}
KOI-12, also known as Kepler-448, is a young two-planet system with an eccentric outer planet KOI-12 c, and an inner Hot Jupiter planet KOI-12 b in a $17.8\,$d orbit~\citep{masuda2017eccentric}. With respect to rotation rate, stellar and planetary radius, the KOI-12/KOI-12 b system is similar to HIP~67522 and TAP~26 (although for the those young stars the radii are large because the stars are still contracting onto the main sequence). Yet, despite its high expected power of interaction, we detect no excess flaring in phase with KOI-12 b. Here, the reason could be that KOI-12 b is in fact super-Alfv\'enic due to the relatively wide orbit compared to TAP 26 and HIP 67522. Intermittent interaction might play a role here as well, and could be favored if future observations capture episodes during which the interaction is 'on'.



%\subsection{Combined Sample}

%For those stars with given transit mid-time and 

\section{Discussion}
\label{sec:discussion}
Eccentricity is almost controlled for in our sample~(Section~\ref{sec:discussion:eccentricity}), while rotation is not well enough constrained to investigate the presence of flaring modulation with $P_{rot}$, or the synodic period~(Section~\ref{sec:discussion:rotsyn}). In the remainder of the discussion, we lay out different reasons for the emerging shape of the distribution in our results in Fig.~\ref{fig:adtest_bp}, in particular the inactive and active branches at high expected power of interaction. We begin with the possible intermittency of the interaction that could be caused by the planet regularly moving in and out of the sub-Alfv\'enic zone for extended periods of time~(Section~\ref{sec:discussion:intermittency}). In Section~\ref{sec:discussion:viewing}, we suggest that the absence of interaction at high expected power could be due to unfavorable viewing geometry of the system, e.g., a face-on orbital plane. In Section~\ref{sec:discussion:misalignment}, we discuss whether (mis-)alignment between spin, orbit and magnetic dipole can increase or decrease the chances of flaring star-planet interaction. In Section~\ref{sec:discussion:commensurability}, we note that systems on the inactive branch are further away from spin-orbit commensurability than stars on the active branch. This leads us to conjecture a possible influence of tidal interactions on the occurrence of orbital phase correlated flaring. We conclude our discussion by following this line of reasoning, and consider ways to disentangle tidal from magnetic interactions~(Section~\ref{sec:discussion:tidalmagnetic}).

\subsection{Eccentricity}
\label{sec:discussion:eccentricity}
If the planetary orbit is highly eccentric, the difference between the semi-major and semi-minor axis can be significantly larger than the variability of the Alfv\'en radius. If then apoastron is outside this radius, and periastron within, the planet may find itself in the sub-Alfv\'enic zone only for a short amount of time, namely during the planet's rapid periastron passage. This was previously exploited in the case of HD 189733, and also for binary interaction in the V 773 Tau system. In our sample, the estimates for eccentricities of the innermost planets are all moderate to low, with $e<XXXX$. The exception is K2-25, with a moderate $e=0.45$, which may contribute to the increased contrast of excess flares. All else equal, closer proximity increases the strength of interaction. Perhaps, K2-25 even moves in and out of the sub-Alfv\'enic zone between periastron and apoastron, turning interaction on and off every single orbit. Otherwise, there are therefore no planets where we would expect a strongly preferred orbital phase for interaction. 


\subsection{Rotational and synodic variability}
\label{sec:discussion:rotsyn}
\citet{fischer2019timevariable} argue that the synodic period, at which the sub-planetary point crosses the same (magnetic) surface element on the star, may result in a clearer interaction signal, because a magnetically active region that is prone to planet-induced flares, will be passed only once per period. However, this assumes that the landscape of active regions on the star is stable throughout the entire observing time, and that the number of such regions in the footpoint passageway is low enough to produce a periodic modulation. 

In our sample, the observing baseline is multiple years long for many systems. Even the most rigidly rotating late M dwarfs show variability in their spot modulation on such long timescales. Even if differential rotation was absent and active regions were extremely stable, we will not be able to observe the passage of the sub-planetary point over the flare-prone region every synodic period. For a fraction of the time, this will happen on the back of the star relative to the observer. The true period will therefore be additionally modulated by the visibility, resulting in an intermittency of the signal. As a consequence, to detect a modulation with the synodic period, a significantly longer observing baseline is needed than to detect a modulation with the orbital period. Additionally, if there are multiple active regions that produce flares when the sub-planetary footpoint passes across, the effect with the synodic period is further diluted. 

In summary: Even for the few stars, for which the rotational period is known precisely enough~(see Fig.~\ref{fig:coherence_hist}), we need to make assumptions about the number of active regions crossed by the footpoint of interaction, differential rotation, spot evolution, and a significant correlation between the location of the planet and the location of the planet-induced flare on the stellar surface. For modulation with the orbital period, we require only that last correlation between footpoint location and orbital phase. Any realistic number of active regions (one or more), a wide range of degrees of differential rotation (from rigid rotator to solar-type), and half-life duration of active regions (day to years) are acceptable, because the observed effect depends predominantly on the visibility of the footpoint, which is governed by orbital motion. And except for extreme circumstances, the latter will be much more stable over long periods of time than the intrinsically dynamic magnetic field configuration. 

\subsection{Intermittency}
\label{sec:discussion:intermittency}
Magnetic star-planet interactions in a given system may sometimes be turned off for episodes of time, and turned on for others. HD 179949 is the prototype example, wherein the orbital modulation of chromospheric indicators was observable for XXXX out of XXXXX epochs~\citep{shkolnik2003evidence,shkolnik2008nature}. It is not clear how long a continuous epoch of interaction or non-interaction might be. This leaves room for many different explanations. Long-term monitoring is needed to investigate the reasons for the temporary cessation of interaction. Candidate mechanisms include: The Alfv\'en radius moving in- and outside the planet's orbit, either over the course of an activity cycle, or through short-term variations in the magnetic field and wind properties caused by, e.g., coronal mass ejections; visibility of the interacting footpoint changing as the magnetic field evolves; the active latitude moving away from the magnetic footpoints passageway, the available magnetic energy decreasing such that the flares or other magnetic interaction indicators still occur but fall below the detection threshold for a period of time.

If the magnetic field of the planet changes or disappears, this might also reduce the intensity of interaction below the detection threshold.

The location of active regions within the passageway of the magnetic footpoints may be such that the interaction occurs only when the region faces away from the observer. This could span multiple orbits of the planet, particularly when the surface is populated by few stable active regions, and the orbital and rotational periods are similar. The flip-flop behavior of active longitudes could then cause a relatively sudden quenching of interaction by moving the interacting region out of view for a period of time. 

In our results in Fig.~\ref{fig:adtest_bp}, we can interpret the two branches emerging at high expected power of interaction as due to the intermittency of the phenomenon. TAP 26 and KOI-12 do not show any sign of excess flaring in our data, despite similar expected powers as in HIP 67522 with its $>2\sigma$ signal. In Sections~\ref{sec:results:individualstars:tap26} and \ref{sec:results:individualstars:koi12}, the absence of interaction in TAP 26 and KOI-12 could be explained by an uncertainty in orbital period that smears out the signal, and a super-Alfv\'enic orbit, respectively. Yet intermittency cannot be ruled out. The case HIP 67522 may be more informative. In our data, we combine the observation from two TESS Sectors that are almost 2 years apart. The four phase-correlated flares all occur in the later Sector, while the earlier one contains only two flares, both of them at orbital phases different from each other and the four phase-aligned flares. This can be interpreted as a tentative sign of intermittency of flaring star-planet interactions in this system. 

\subsection{Viewing geometry}
\label{sec:discussion:viewing}
The visibility of the magnetic interaction footpoint is crucial for the detection of flaring star-planet interactions in our analysis. The viewing geometry of the star-planet system is therefore important to consider. Assuming for the sake of the argument that the sub-planetary point is close to the magnetic interaction footpoint in longitude, two factors will determine whether the footpoint's visibility will be modulated with the planetary orbit or not: the orbital inclination and the footpoint's latitude relative to it. If the footpoint's latitude is low, i.e., close to the orbital plane, the orbital plane can be inclined to a high degree, and the footpoint will still periodically move in and out of view. However, if the footpoint's latitude is high, we need to observe the star-planet system nearly edge-on to be able to measure a modulation of activity with the planet's orbital phase. The latter can be the case when the dipole axis of the stellar large scale field is well-aligned with the orbital axis. In that case, assuming the dipole field dominates at the distance of the planet, the footpoint will be close to the pole, similar to the hot spot created by Io and the other moons in the polar cap of Jupiter. In this scenario, many non-transiting systems may experience flaring star-planet interaction, but those would not show up as correlation of flare timing with the orbital phase in our data.

Most of our systems are transiting, and for those we can expect this scenario to be a minor problem. Of the systems that have high expected power but do not show any, only TAP 26 is a planet detected in radial velocity, and is expected to have a relatively high orbital inclination of about $55\pm10$ deg inferred from Zeeman Doppler measurements~\citep{yu2017hot}. The ZD map reveals magnetic field concentrations close to the rotational pole. If the orbital plane is aligned with the stellar rotation axis, the absence of phase-correlated flaring in TAP 26 could be explained by the footpoint of interaction being always in view and therefore never modulated.

\subsection{(Mis-)alignment between orbital, rotational and stellar magnetic axis}
\label{sec:discussion:misalignment}
Another relevant factor is the geometry of the system. If we assume that the magnetic dipole axis and the rotation axis of the star are relatively well-aligned, the spin-orbit alignment may have an influence on interaction strength. 


or more broadly system architecture including misalignment of the magnetic axis. See HAT-P-11, or instance, and the many polar orbits in \cite{bourrier2023dream}

\cite{miskovetz2022resolving}  found background stars that were not co-moving HAT-P-11. HAT-P-11 rotates nearly pole-on, and HAT-P-11 b is transiting in a near-polar orbit~\citep{bourrier2023dream}. This may be the reason for absent SPI.

HD 179949 has a strong misalignment of about 70 deg between rotational and magnetic axis. Time-resolved X-ray observations~\citep{acharya2022xray} suggest that the magnetic interaction footpoint is close to the stellar rotational pole, which would be close to the magnetic equator of the strongly dipolar field of HD 179949



\subsection{Spin-orbit commensurability}
\label{sec:discussion:commensurability}
In Fig.~\ref{fig:spinorbit}, we split our sample into three categories. The first contains the systems where the expected interaction is lower than that in TOI-540. The second includes the systems where a high power of interaction is expected, but none is measured (e.g., TAP 26, KOI-12). The final third category contains all system where the data suggest a correlation between expected power of interaction and measured flaring star-planet interaction signal (e.g., HIP 67522, AU Mic). We search for spin-orbit commensurability with integers $<10$ in each system, and plot the relative difference between the closest ratio and the actual spin-orbit ratio. Systems with low expected power of interaction span both very close spin-orbit commensurability to none. 
Systems with high expected power are divided: Stars that show a correlation between excess flaring and expected power tend to have closer spin-orbit commensurability than systems that are far away from any low-integer ratio between rotational and orbital period.

While chance alignment is possible, the division invokes the idea that spin-orbit commensurability and magnetic star-planet interaction are related. Either something about our observing method favors the measurement of excess flaring in systems with close spin-orbit commensurability, or there is physical causation at play. In the latter case, the spin-orbit commensurability could be either an effect, or a cause, of measured magnetic star-planet interaction.

Our method favors systems in which the deviation from uniform flare timing is concentrated in a narrow phase bracket of the planet's orbit. This might be the case in HIP 67522. Due to its low-order, 5:1 resonance, the sub-planetary point passes the same magnetic features at the same orbital phase each orbit. This is no longer the case for m:n ratios, where $n>1$. In those systems, the same surface feature is passed by the sub-planetary point at the same orbital phase every $m \cdot n$ orbits, if $m>n$ (which is true for all our systems with high expected power of interaction). For HIP 67522, the interaction signal may therefore be more strongly confined to certain orbital phases, while for systems with less and less commensurability, this effect dilutes. 

There could also be physical reasons for increased flaring interaction-in systems with close spin-orbit commensurability.  \citet{lanza2022model} suggest that tidal torque of the planet on the star excites resonant oscillations in the interior stellar magnetic field, which in turn leads to a resonance between stellar spin and planetary orbit. These oscillations can lead to a surface pattern of emerging magnetic field that rotates with the spin-orbit commensurable period. This can explain the clearer interaction signal in the systems with spin-orbit resonance, because it would stabilize the pattern of regions with strong magnetic field that favor planet-induced flaring, and thereby counterbalance the diluting effect of differential rotation. This model requires tidal interaction, which may be too weak in systems like AU Mic, where the planet is too small and distant for the proposed effect~\citet{lanza2022model}, but that still show spin-orbit commensurability and signs of flaring star-planet interaction. 
However, for systems with $m=1$, like HIP~67522, or $\tau$ Boo~\citep[which is not in our sample]{walker2008nature, cauley2019magnetic}, this model would directly favor active regions that rotate with the orbital period instead of the rotational one.

While in~\citet{lanza2022model}, resonant oscillations favor flaring star-planet interaction observability, there might also be physical effects where the magnetic torque causes the spin-orbit resonance in the first place. The connection here is: Where there is strong enough magnetic torque to cause spin-orbit resonance, there would also be flaring star-planet interaction. 

\citet{szabo2021changing} note that many systems within the sparsely populated region of close-in planets around rapid rotators~\citep{mcquillan2013stellar} seem close to low-integer spin-orbit resonance. The systems with consistently high expected power of interaction (TAP 26, TOI-540, KOI-12, HIP 67522, K2-25, and AU Mic) are all found within that region, even if KOI-12 and AU Mic lie rather close to its edge. Of these six systems, those that show signs of star-planet interaction, are also in close-spin orbit resonance. We may speculate, that at least some of the mechanisms that explain the dearth of close-in planets around rapidly rotating stars also explain the presence of star-planet magnetic interaction in those systems that are still found there. One may think that planets that remain in the desert are held there by being trapped in the resonance, while for the others, the magnetic torque cause rapid migration out of this region.


\begin{figure}[t]
    \script{paper_spin_orbit_commensurability.py}
    \begin{centering}
        \includegraphics[width=\linewidth]{figures/PAPER_spin_orbit_commensurable.png}
        \caption{
           Spin-orbit commensurability compared to flaring star-planet interaction signal. The lower the p-value of the AD test, the stronger the interaction signal. Lower relative difference between the measured spin-orbit ratio and the integer ratio closest to it indicates a tighter spin-orbit commensurability. We include only ratios of integers $<10$. \textbf{Black diamonds:} All systems with p-values lower than that of K2-25, and expected power of interaction equal or higher than TOI-540's, the active 'SPI on' branch~(see top left panel in Fig.~\ref{fig:adtest_bp}). \textbf{Red crosses:} The complementary 'SPI off' branch includes all systems with expected power of interaction above that of TOI-540, but with p-values below that of K2-25. \textbf{Blue circles:} The remaining systems include all with expected power of interaction below that of TOI-540. Among the systems with high expected power, those with higher interaction signal ('SPI on') show closer spin-orbit commensurability than those on the inactive 'SPI off' branch.
        }
        \label{fig:spinorbit}
    \end{centering}
\end{figure}

\subsection{Tidal or magnetic interaction?}
\label{sec:discussion:tidalmagnetic}
The tight spin-orbit commensurability in some systems may give rise to stable spot patterns on the star that rotate with the commensurable period~\citep{lanza2022model}, and perhaps appear preferentially in the orbital plane of the tidally interacting planet. It is possible that the enhanced periodicity seen with the orbital period in systems close to a low integer spin-orbit commensurability is an effect of such tidally created spots that lie along the passageway of the planet's magnetic footpoint.% On the other hand, magnetic interactions, wherein the direct perturbation of stellar field by the planetary obstacle cannot be ruled out~\citep{lanza2018closeby, saur2013magnetic}.

Both mechanisms may play a role, but to understand their respective contribution, we require approaches that can unambiguously discriminate between the two.


While suggestive, our data are statistically insufficient to conclude whether the flaring interaction is driven only by magnetic interaction, or whether tidal effects, and more specifically spin-orbit resonances, play a role. Further monitoring of the systems in our sample will clarify whether the interaction is statistically significant for individual systems, or whether the branching is caused by intrinsic intermittency of the effect. If the branching persists over time, and systems do not switch from the inactive to the active branch or vice versa, the distinction may more likely be due to spin-orbit resonance.

Another possibility is that the branching disappears in favor of a continuous distribution. This would put in question the on-off nature of magnetic interactions.


Ways to investigate which mechanisms play a role using flaring:

Search for flaring star-planet interactions in rapidly rotating fully convective stars with close-in planets or brown dwarfs. There are very few such systems, TOI-540 among them. These systems have only negligible differential rotation so that the rotation period of surface features can be precisely known. In those cases, we may figure out whether interaction occurs with the synodic period.

Search for ECMI emission in radio from systems that have high expected power of interaction. If periodic radio emission is seen predominantly in the systems that also show flaring star-planet interactions, magnetic interaction is relevant because coherent radio emission stems from large scale fields, and not from resonantly emerging patterns of active regions as suggested by~\citet{lanza2022model}.

spetral type dependence of tidal interaction

\section{Summary and Conclusions}
\label{sec:summary}

\bibliography{bib}

\appendix
\section{Notes on individual stars}
\subsection{Proxima Cen}
Although Proxima Cen is part of a triple stellar system, we treat it a single star in this work because of the large angular separation to its two companions. Proxima Cen b has an eccentricity upper limit of 0.35~\cite{anglada-escude2016terrestrial}.
\subsection{TAP 26}
Likely low eccentricity given in \cite{yu2017hot}.

\subsection{GJ 393}
Eccentricity likely low, but not quantified in \cite{amado2021carmenes}, so we treat it as unconstrained (set e=0.5)

\subsection{Kepler-42}
 KOI-961 in the detection paper \cite{muirhead2012characterizing}, low eccentricity constrained by \cite{mann2017gold}


\subsection{K2-354}
We note that we found K2-354 under EPIC 211314705 and K2-329(!) in \cite{bouma2020cluster}, who refer to the detection paper \cite{pope2016transiting}, and also under TIC 468989066.
\end{document}



% Define document class
\documentclass[twocolumn]{aastex631}
\usepackage{showyourwork}

% Begin!
\begin{document}

% Title
\title{Searching for flaring star-planet interactions in star-planet systems in Kepler and TESS}

% Author list
\author{Ekaterina Ilin}
\author{Katja Poppenh\"ager}

% Abstract with filler text
\begin{abstract}
    Star-planet system we have found so far are nothing like the Solar System. Many stars host planets in orbits much closer than Mercury, where it takes them mere days to complete a full revolution. In some of these systems, the planets are so close that they plow through the extended magnetosphere of the star, and are expected to trigger flares on their surfaces. So far, however, these flaring star-planet interactions escape our detection methods. If found, the planets and the flare they cause become localized probes of the parts of the stellar magnetosphere that extends from the location of one to the other. A measurement of these interactions constrains mass loss from the star, and potentially also the magnetic field of the planet. 

    In this work, we searched star-planet systems observed by Kepler and TESS for flaring star-planet interactions. We compiled a catalog of all flaring star-planet systems known to date, and characterized the detected flares detected. We then searched for flares in excess of the intrinsic flaring of the star, i.e., flares triggered by the interaction with the planet in phase with its orbit. 

    We find XXXX star-planet systems with > XXXX flares. For XXXX of them, we can calculate the expected amplitude of interaction based on the system's magnetic field strength, orbital parameters, rotation and planet radius. We find that the trend of higher interaction signal agrees with the expected strength, but also shows signs of intermittency, both for individual systems, and in comparison between ones with similar expected amplitude. HIP 67522, a 17 Myr young Sun with a Jupiter in a 9 day orbit, shows both one of the highest expected and the highest measured interaction amplitudes, albeit only at the $2.5\sigma$ level. 
\end{abstract}

% Main body with filler text
\section{Introduction}
\label{sec:intro}
Star-planet system we have found so far are nothing like the Solar System.

What is SPI?

Why look for SPI? Use planets as probes stellar magnetic fields at a defined orbit, e.g., stellar wind conditions. SPI can only occur if the planet is orbiting within the Alfv\'en radius of the star for a sizable period of time. The Alfv\'en radius is the radius at which the stellar wind velocity, which increases with increasing distance from the star because of XXXX, exceed the Alfv\'en velocity of the magnetized plasma. Beyond this radius, this plasma is disconnected from the star. Alfv\'en waves cannot propagate back to the star because the wind is carrying the plasma away faster than these waves can travel.





\section{Data}
\label{sec:data}
\subsection{Star-Planet Systems}
%Catalogs
% as of January 2021
We started our sample selection with the Planetary Systems Composite Parameters Table (PSCP)\footnote{ \url{https://exoplanetarchive.ipac.caltech.edu/cgi-bin/TblView/nph-tblView?app=ExoTbls&config=PSCompPars}} from which we removed controversial planet detections (``pl\_controv\_flag'' set to 1), i.e. detections with existing literature questioning the result. The catalog then contained XXXX unique systems.

We complemented the PSCP with the TESS TOI Table~\citep[TT][]{guerrero2021arxiv}\footnote{\url{https://tev.mit.edu/data/collection/193/}}. We filtered the catalog by requiring the confirmed planet (CP) or known planet (KP) flags to be set in the ``TOI Disposition''. This added XXXX star-planet systems to the sample.
\subsection{Multiplicity}
We flag all multiple stars except for Proxima Cen (whose companion are well separated both physically and on the sky), but kept them in for the analysis. In addition to noting the multiplicity of some well-known stars, we also use the Robo-AO catalog of nearby companions to Kepler planet candidates from ~\cite{ziegler2018measuring}


\subsection{Orbital periods}
Either Kepler/TESS or from RVs.

\subsection{Semi-major axes}
\label{sec:data:a}
The critical parameter is the distance between the star and the planet, not the semi-major axis itself. This varies if the orbit is eccentric. To account for varying distance, we adopt the mean semi-major axes from the literature, but use a custom estimate for the uncertainty that includes the eccentricity.
If the eccentricity is known, then the range of the distance is set to either the error on the semi-major axis or half of the difference between the semi-major and semi-minor axis, as calculated from the eccentricity -- whichever is larger. If eccentricity is NOT known, then the range of the distance is either the 25\% error on the semi-major axis, that is, assuming e=0.5 or the uncertainty on the semi-major axis -- whichever is larger.

\subsection{Rotation periods}
We adopt available rotation period values from the literature. When uncertainties are not given, we assume 10\% uncertainty.% _09_

\subsection{Planetary radii}
NASA table, see references in table, or using the planetary mass to calculate the radius using the empirical relations derived in \cite{chen2007} using their open source \texttt{forecaster} tool, upgraded to \texttt{astro-forecaster}\footnote{https://pypi.org/project/astro-forecaster/} by Ben Cassesse. 

\subsection{Relative velocity}

We calculate the relative velocity between the magnetic field of the star, assumed to be co-rotating, at the planetary orbit, as

\begin{equation}
    v_{rel} = 2 \pi a \left(\dfrac{1}{P_{orb}} - \dfrac{1}{P_{rot}}\right) 
\end{equation}

We use quadratic error propagation to estimate the uncertainty in $v_{rel}$ with the mean of the upper and lower uncertainty values on the orbital period and rotation period, if both are given, and the uncertainty on semi-major axis as derived in Sec.~\ref{sec:data:a}. 



%Transit information
%Full sample statistics
\subsection{Kepler and TESS photometry}
Between 2009 and 2013, the Kepler space telescope~\citep{koch2010kepler} nearly continuously observed a patch of the sky in the Cygnus-Lyra region. Each of the 18 observing Quarters contains nearly uninterrupted $\sim 90$ days of observations, totaling over 100000 stars monitored in 2-min cadence. 
% 100000 from here: https://exoplanetarchive.ipac.caltech.edu/docs/KeplerMission.html

The Transiting Exoplanet Survey Satellite~(TESS,~\cite{ricker2015transiting}) is an all-sky mission that began operations in 2018, completed two sky scans until summer 2022. It is observing at the time of writing, collecting nearly continuous photometric time series in the 600-1000 nm band for $\sim 27 days$ in each observing Sector. About $200\,000$ stars have been observed in 2-min cadence in the first two years of operations with about 20000 targets per Sector. Out of these, from Sector 27 on, 1000 targets were observed at even higher 20-s cadence in each Sector. 
% 20000 per sector does not add up, because of stars observed in multiple sectors, the 20000 and 1000 figures can be found here: https://tess.mit.edu/observations/target-lists/
% extended TESS mission https://heasarc.gsfc.nasa.gov/docs/tess/extended.html

Based on the filtered PSCP and TT tables, we queried the full Kepler archive (quarters Q0-Q17, Data Release 25) and most recent TESS catalog (July 2022) for their respective 1-min and 2-min/20-s cadence light curves, accounting for systems that appeared in both catalogs. In total, we obtained XXXX Kepler light curves and XXXX TESS light curves. XXXX systems were observed only by the primary Kepler mission, XXXX only by TESS, and XXXX by both missions.

\subsection{X-ray luminosity}
We take the \cite{foster2022exoplanet} X-ray luminosity catalog as the basis for our calculation, assuming an uncertainty of 30\% in the reported values, covering typical intrinsic variability among stars of similar rotation periods in studies like ~\cite{wright2011stellaractivityrotation, wright2018stellar}.

\subsection{Stellar magnetic fields}
\cite{foster2022identifying} and \cite{reiners2022magnetism} 

The intrinsic scatter in Eq. 4 in \cite{reiners2022magnetism} dominates the uncertainty in B over the uncertainty in stellar radius and $L_X$. We use the $+/-1\sigma$ values for the latter two for the final uncertainty estimate $\sigma B$

The estimate is consistent with existing Zeeman broadening for, e.g. AU Mic (3010 G~\cite{reiners2022magnetism})
\subsection{Transit mid-times}

\section{Methods}
\label{sec:methods}
We measure flaring star-planet interactions (SPI) as the presence of excess flares triggered by the orbiting planet. These excess flares manifest as a deviation of flare timing from a random Poisson process with orbital phase. That means that in the absence of flaring SPI, flare peak time will be distributed randomly in orbital phase. In the presence of flaring SPI, we measure a phase dependent deviation from this randomness.

The main data for this analysis are flare times. To obtain them, we gather the Kepler and TESS light curves for all star-planet systems, remove rotational variability trends, search the de-trended light curves for flares, and convert the flare times to orbital phases. We then perform a customized Anderson-Darling test on the flare peak time distribution, which yields a p-value for the significance of the SPI signal. We compare these results to theoretical SPI power $P_{SPI}$ using the scaling relations in~\cite{lanza2012starplanet}, where $P_{SPI}$ is proportional to

\begin{equation}
    P_{SPI} \sim B_p^{2/3} B_*^{4/3} v_{rel} R_p^2.
\end{equation}

$B_p$ and $B_*$ are the total planetary and stellar surface field strengths, respectively; $v_{rel}$ is the relative velocity between the stellar rotation and the planet's orbit at the semi-major axis; and $R_p$ is the planetary radius.


The methods for light curve de-trending and flare finding in Kepler and TESS light curves, as well as the Anderson-Darling test, are the same as detailed in~\citep{ilin2022searching}. We therefore only briefly recap the techniques here:

To remove trends and rotational variability from the light curve without losing the flare signal, we use an empirically derived multistep process. First, we apply spline fit with a coarse sampling of 30h mean values to capture slow rotation with periods above multiple days and non-periodic trends. Then, we iteratively fit a series of sines to capture rotational signal on time scales down to about half a day, close to the fastest rotational signals measured in low mass stars. Eventually, we apply two Savitzky-Golay filters in sequence, with decreasing periods at each step. At this stage, we mask all data points above a 2.5 sigma (or 1.5 sigma for very active stars) threshold as flare candidates to prevent the filter from ironing out the flares.

Flare finding here. Manual vetting

Orbital phase finding. Check coherence times. Orbital period is enough.
Use transit mid-times from the same mission if possible.

Anderson-Darling test here. Used four equidistant start phases instead of 20, because it does not change much.

\cite{foster2022identifying} and \cite{reiners2022magnetism} to get Lx and then B.

\section{Results}
\label{sec:results}

\subsection{Flare catalog}

\begin{table*}
    \script{paper_latex_flare_table.py}
    \centering
            \caption{
            Flare catalog of all star-planet systems observed by Kepler and TESS. In transiting multi-planet systems, the orbital phase refers to the innermost planet, with the transit mid-time at phase zero. 
        }
    \input{output/flare_table.tex}
        \label{fig:random_numbers}
\end{table*}

\subsection{Period coherence times}

divide the coherence time of the orbital period by the observing baseline of the combined Kepler and TESS observations for each star.

Do the same for the mid-times.

\begin{figure}[ht!]
    \script{paper_coherence_histogram.py}
    \begin{centering}
        \includegraphics[width=\linewidth]{figures/PAPER_coherence_histogram.png}
        \caption{
           Time span of observation vs. coherence time or the orbital and rotational periods, respectively. Orbital periods are all known precisely enough, so that the phase uncertainty at the last observed flare is of the order of $10^{-2}$. In contrast, in stars with known rotation periods, the precision is often too low, so that the phase of the last flare is often undetermined (ratio on x-axis $\sim 1$.
        }
        \label{fig:coherence_hist}
    \end{centering}
\end{figure}

\subsection{Flare phase distributions}

For details on AU Mic results, see~\citep{ilin2022searching}.

\begin{figure*}[ht!]
    \script{paper_cumdist_individual.py}
    \begin{centering}
        \includegraphics[width=\linewidth]{figures/PAPER_flares_phase_hist.png}
        \caption{
            Cumulative distributions of orbital phases of flares in the hosts observed by Kepler and TESS that had the most flares detected per star. 
        }
        \label{fig:cumdist_active}
    \end{centering}
\end{figure*}


\subsection{Flaring star-planet interaction signal}
We use the custom Anderson-Darling test detailed in \cite{ilin2022searching} to look for flaring SPI in all system with three flares or more. We do not find $>3\sigma$ signal of star-planet interaction in any of the star-planet systems in this sample.  



\begin{table*}
    \script{paper_main_table.py}
    \centering
            \caption{
           Flaring SPI results
        }
    \input{output/fit_parameters_bibkeys.tex}
        \label{tab:maintable}
\end{table*}


% \begin{figure*}[ht!]
%     \script{paper_adtext_vs_value_scatterplots.py}
%     \begin{centering}
%         \includegraphics[width=\linewidth]{figures/PAPER_ADtest_vs_p_spi_erg_s_bp0.png}
%         \caption{
%             Expected power of SPI vs. AD test results, assuming an unmagnetized planet. These results are mainly dominated by the X-ray luminosity of the host star, see Fig.~\ref{fig:adtest_xray_flux_erg_s}. Assuming a magnetized planet with $B_p=1$ G has only minor influence on the results shown in this figure, see Appendix.
%         }
%         \label{fig:adtest_p_spi_erg_s_bp0}
%     \end{centering}
% \end{figure*}


% \begin{figure*}[ht!]
%     \script{paper_adtext_vs_value_scatterplots.py}
%     \begin{centering}
%         \includegraphics[width=\linewidth]{figures/PAPER_ADtest_vs_xray_flux_erg_s.png}
%         \caption{X-ray luminosity of the host star~\cite{foster2022identifying} vs. AD test results.
%         }
%         \label{fig:adtest_xray_flux_erg_s}
%     \end{centering}
% \end{figure*}

\subsection{Individual stars}

\subsubsection{Proxima Cen}
tentative detection of another planet further in, at 0.029 au, or 5 day orbital period~\citep{faria2022candidate, artigau2022linebyline}
Although Proxima Cen is part of a triple stellar system, we treat it a single star here because of the large angular separation to its two companions.

Proxima Cen b has an eccentricity upper limit of 0.35~\cite{anglada-escude2016terrestrial}

\subsubsection{HIP 67522}
HIP 67522 has one of the strongest expected SPI signals in our sample, and shows the clearest signs of flaring SPI, albeit with only 6 flares in the sample. These flares are distributed across two Sectors in TESS. In Sector 11, two flares occur at orbital phases $\sim 0.6$ and $\sim 0.8$. Two years later, in Sector 38, four flares occur, but this time, all of them take place at orbital phases $0.00-0.05$, that is, up to a few hours after transit.

HIP 67522 is young Sun, currently contracting onto the main sequence. It is a Sco-Cen member (10-20 Myr old), which was discovered to host a close-in Jupiter, HIP 67522 b, in 2020~\citep{rizzuto2020tess}. It was found with low obliquity, consistent with XXXX~\citep{heitzmann2021obliquity}. Tentative third planet transiting.

HIP 67522 is in close spin-orbit commensurability -- $P_{rot}/P_{orb}=5/1$. The system's $a/R_*\approx11.7$, eccentricity below .25~\cite{rizzuto2020tess}, so that  

\cite{wood2021characterizing} rule out stellar companions for HIP 67522 based on RV, high-resolution imaging, and Gaia data.

HIP 67522 will be observed again with TESS in spring 2023.
\subsubsection{GJ 3323}

GJ 3323 b is a non-transiting, likely rocky planet. We adopted the mean radius between assuming pure iron through pure ice composition in \cite{lovos2022null} Table A1, and took the difference from this mean to either of the extremes as uncertainty estimate

\subsubsection{GJ 1061}
GJ 1061 b is a non-transiting, likely rocky planet. We adopted the mean radius between assuming pure iron through pure ice composition in \cite{lovos2022null} Table A1, and took the difference from this mean to either of the extremes as uncertainty estimate

\subsubsection{TAP 26}
Likely low eccentricity given in \cite{yu2017hot}

\subsubsection{GJ 393}
Eccentricity likely low, but not quantified in \cite{amado2021carmenes}, so we treat it as unconstrained (set e=0.5)

\subsubsection{Kepler-42}
 	KOI-961 in the detection paper \cite{muirhead2012characterizing}, low eccentricity constrained by \cite{mann2017gold}

\subsubsection{GJ 674}
Has hot UV flares~\cite{froning2019hot}
The mass estimates in \cite{bonfils2007harps} and \cite{boisse2011disentangling} do not quote errors, but differ by 0.3MEarth, so we assume that as the uncertainty on $M_p\sin i$

\subsubsection{K2-354}
Found under EPIC 211314705 and K2-329 in \cite{bouma2020cluster}, who refer to the detection paper \cite{pope2016transiting}, also TIC 468989066.

\subsubsection{DS Tuc A}
It's a multiple
\subsubsection{Kepler-1651}
It's a multiple


\subsubsection{Excluded flaring star-planet systems}
\paragraph{LTT 1445}
 LTT 1445 is a hierarchical triple wherein LTT 1445 A hosts two close-in planets, and LTT 1445 B and C are a tight binary system separated by about 7 arcsec, that is, blended in one TESS pixel~\citep{winters2019three}. Both the binary and LTT 1445 A have been detected with flares in X-ray~\citep{brown2022xray}. Recently, LTT 1445 A was found to host a second planet LTT 1445 Ac inwards of LTT 1445 Ab~\citep{lavie2022planetary}. The rotation periods are not known, so it does not appear in the final figure.

\paragraph{HD 41004 B}
Hierarchical system, composed of HD 41004 A and its Jupiter HD 41004 Ab, and HD 41004 B. an M dwarf, and its brown dwarf companion HD 41004 Bb. The rotation periods are not well-known, so it does not appear at all in the final figure.


\subsection{Combined Sample}

For those stars with given transit mid-time and 
\section{Discussion}
\label{sec:discussion}

\subsection{Similar work that puts ours in context}

\subsection{Rotational variability}
Hard to measure, because it de-phases much faster than orbital variability due to differential rotation and spot evolution. We could use single light curves within which coherence is given...

\subsection{Intermittency}

TAP 26 vs. HIP 67522, also different sectors for HIP 67522, see \cite{shkolnik2008nature}

\subsection{Spin-orbit alignment}

\bibliography{bib}

\appendix
\section{Expected $P_{SPI}$ vs. measure flaring SPI with a magnetized planet}

\begin{figure*}[ht!]
    \script{paper_adtext_vs_value_scatterplots.py}
    \begin{centering}
        \includegraphics[width=\linewidth]{figures/PAPER_ADtest_vs_p_spi_erg_s_sab_wBp_colorcode_dist.png}
        \caption{
            Expected power of SPI from stretch and break vs. AD test results, assuming a magnetized planet with $B_p=1$ G. 
        }
        \label{appendix:fig:adtest_p_spi_erg_s}
    \end{centering}
\end{figure*}

\end{document}


